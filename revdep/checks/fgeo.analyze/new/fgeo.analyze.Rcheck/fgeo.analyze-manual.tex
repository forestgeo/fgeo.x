\nonstopmode{}
\documentclass[a4paper]{book}
\usepackage[times,inconsolata,hyper]{Rd}
\usepackage{makeidx}
\usepackage[utf8]{inputenc} % @SET ENCODING@
% \usepackage{graphicx} % @USE GRAPHICX@
\makeindex{}
\begin{document}
\chapter*{}
\begin{center}
{\textbf{\huge Package `fgeo.analyze'}}
\par\bigskip{\large \today}
\end{center}
\begin{description}
\raggedright{}
\inputencoding{utf8}
\item[Title]\AsIs{Analyze ForestGEO Data}
\item[Version]\AsIs{1.1.3}
\item[Description]\AsIs{Analyze ForestGEO data
(<http://www.forestgeo.si.edu/>).}
\item[License]\AsIs{GPL-3}
\item[URL]\AsIs{}\url{https://github.com/forestgeo/fgeo.analyze}\AsIs{}
\item[BugReports]\AsIs{}\url{https://github.com/forestgeo/fgeo.analyze/issues}\AsIs{}
\item[Depends]\AsIs{R (>= 3.2)}
\item[Imports]\AsIs{dplyr (>= 0.7.8), fgeo.tool (>= 1.2.1), glue (>= 1.3.0),
graphics, grDevices, lubridate (>= 1.7.4), magrittr (>= 1.5),
MASS, purrr (>= 0.3.0), rlang (>= 0.3.1), stats, tibble (>=
2.0.1), tidyr (>= 0.8.2), withr (>= 2.1.2)}
\item[Suggests]\AsIs{covr (>= 3.2.1), fgeo.x (>= 1.1.0), ggplot2 (>= 3.1.0), knitr
(>= 1.21), measurements (>= 1.3.0), readr (>= 1.3.1), rmarkdown
(>= 1.11), spelling (>= 2.0), testthat (>= 2.0.1)}
\item[Additional\_repositories]\AsIs{https://forestgeo.github.io/drat/}
\item[Encoding]\AsIs{UTF-8}
\item[Language]\AsIs{en-US}
\item[LazyData]\AsIs{true}
\item[Roxygen]\AsIs{list(markdown = TRUE)}
\item[RoxygenNote]\AsIs{6.1.1}
\item[NeedsCompilation]\AsIs{no}
\item[Author]\AsIs{Mauro Lepore [aut, ctr, cre],
Gabriel Arellano [aut, rev],
Richard Condit [aut],
Matteo Detto [aut],
Kyle Harms [aut],
Suzanne Lao [aut, rev],
KangMin Ngo [rev],
Haley Overstreet [rev],
Sabrina Russo [aut, rev],
Daniel Zuleta [aut, rev],
CTFS-ForestGEO [cph]}
\item[Maintainer]\AsIs{Mauro Lepore }\email{leporem@si.edu}\AsIs{}
\end{description}
\Rdcontents{\R{} topics documented:}
\inputencoding{utf8}
\HeaderA{abundance}{Abundance and basal area, optionally by groups.}{abundance}
\aliasA{basal\_area}{abundance}{basal.Rul.area}
%
\begin{Description}\relax
\begin{itemize}

\item \code{\LinkA{abundance()}{abundance}} counts the number of rows in a dataset, optionally by groups
created with \code{\LinkA{dplyr::group\_by()}{dplyr::group.Rul.by()}} (similar to \code{\LinkA{dplyr::n()}{dplyr::n()}}). It warns if it
detects duplicated values of treeid.
\item \code{\LinkA{basal\_area()}{basal.Rul.area}} sums the basal area of
all stems in a dataset, optionally by groups created with \code{\LinkA{group\_by()}{group.Rul.by}}. It
warns if it detects duplicated values of stemid. It does not convert units
(but see examples).

\end{itemize}


Both \code{\LinkA{abundance()}{abundance}} and \code{\LinkA{basal\_area()}{basal.Rul.area}} warn if they detect
multiple censusid and multiple plots.
\end{Description}
%
\begin{Usage}
\begin{verbatim}
abundance(data)

basal_area(data)
\end{verbatim}
\end{Usage}
%
\begin{Arguments}
\begin{ldescription}
\item[\code{data}] A dataframe. \code{\LinkA{basal\_area()}{basal.Rul.area}} requires a column named \code{dbh} (case
insensitive).
\end{ldescription}
\end{Arguments}
%
\begin{Details}\relax
You may want to calculate the abundance or basal area for a specific subset
of data (e.g. "alive" stems or stems which \code{dbh} is within some range).
Subsetting data is not the job of these functions. Instead see
\code{\LinkA{base::subset()}{base::subset()}}, \code{\LinkA{dplyr::filter()}{dplyr::filter()}}, or \code{[}.
\end{Details}
%
\begin{SeeAlso}\relax
\code{\LinkA{dplyr::n()}{dplyr::n()}}, \code{\LinkA{dplyr::group\_by()}{dplyr::group.Rul.by()}}.

Other functions for abundance and basal area: \code{\LinkA{abundance\_byyr}{abundance.Rul.byyr}}
\end{SeeAlso}
%
\begin{Examples}
\begin{ExampleCode}
library(fgeo.tool)

# abundance() -------------------------------------------------------------

abundance(data.frame(1))

# One stem per tree
tree <- tribble(
  ~TreeID, ~StemID, ~DBH,
  "1", "1.1", 11,
  "2", "2.1", 21
)

abundance(tree)

# One tree with multiple stems
stem <- tribble(
  ~TreeID, ~StemID, ~DBH,
  "1", "1.1", 11,
  "1", "1.2", 12
)

abundance(stem)
## Not run: 
# Similar but more realistic
assert_is_installed("fgeo.x")
stem <- fgeo.x::download_data("luquillo_stem5_random")

abundance(stem)

abundance(pick_main_stem(stem))

## End(Not run)

vft <- tribble(
  ~PlotName, ~CensusID, ~TreeID, ~StemID, ~DBH,
  "p", 1, "1", "1.1", 10,
  "q", 2, "1", "1.1", 10
)

# * Warns if it detects multiple values of censusid or plotname
# * Also warns if it detects duplicated values of treeid
abundance(vft)

# If trees have buttressess, the data may have multiple stems per treeid or
# multiple measures per stemid.
vft2 <- tribble(
  ~CensusID, ~TreeID, ~StemID, ~DBH, ~HOM,
  1, "1", "1.1", 88, 130,
  1, "1", "1.1", 10, 160,
  1, "2", "2.1", 20, 130,
  1, "2", "2.2", 30, 130,
)

# You should count only the main stem of each tree
(main_stem <- pick_main_stem(vft2))

abundance(main_stem)

vft3 <- tribble(
  ~CensusID, ~TreeID, ~StemID, ~DBH, ~HOM,
  1, "1", "1.1", 20, 130,
  1, "1", "1.2", 10, 160, # Main stem
  2, "1", "1.1", 12, 130,
  2, "1", "1.2", 22, 130 # Main stem
)

# You can compute by groups
by_census <- group_by(vft3, CensusID)
(main_stems_by_census <- pick_main_stem(by_census))

abundance(main_stems_by_census)

# basal_area() ------------------------------------------------------------

# Data must have a column named dbh (case insensitive)
basal_area(data.frame(dbh = 1))

# * Warns if it detects multiple values of censusid or plotname
# * Also warns if it detects duplicated values of stemid
basal_area(vft)

# First you may pick the main stemid of each stem
(main_stemids <- pick_main_stemid(vft2))

basal_area(main_stemids)

# You can compute by groups
basal_area(by_census)
## Not run: 
measurements_is_installed <- requireNamespace("measurements", quietly = TRUE)
if (measurements_is_installed) {
  library(measurements)

  # Convert units
  ba <- basal_area(by_census)
  ba$basal_area_he <- conv_unit(
    ba$basal_area,
    from = "mm2",
    to = "hectare"
  )

  ba
}

## End(Not run)

\end{ExampleCode}
\end{Examples}
\inputencoding{utf8}
\HeaderA{abundance\_byyr}{Create tables of abundance and basal area by year.}{abundance.Rul.byyr}
\aliasA{basal\_area\_byyr}{abundance\_byyr}{basal.Rul.area.Rul.byyr}
%
\begin{Description}\relax
\begin{itemize}

\item \code{abundance\_byyr()} first picks the main stem of each tree (see
?\code{\LinkA{fgeo.tool::pick\_main\_stem()}{fgeo.tool::pick.Rul.main.Rul.stem()}}). Then, for each species and each
round-mean-year of measurement, it counts the number of trees. The result
includes \strong{main stems} within a given dbh range.
\item \code{basal\_area\_byyr()} first sums the basal basal area of all stems of each
tree. Then, for each species and each round-mean-year of measurement,
it sums the basal area of all trees. The result includes all stems within a
given dbh range (notice the difference with \code{abundance\_byyr()}).

\end{itemize}

\end{Description}
%
\begin{Usage}
\begin{verbatim}
abundance_byyr(vft, ...)

basal_area_byyr(vft, ...)
\end{verbatim}
\end{Usage}
%
\begin{Arguments}
\begin{ldescription}
\item[\code{vft}] A ForestGEO-like dataframe; particularly a ViewFullTable. As such,
it should contain columns \code{PlotName}, \code{CensusID}, \code{TreeID}, \code{StemID},
\code{Status}, \code{DBH}, \code{Genus}, \code{SpeciesName}, \code{ExactDate}, \code{PlotCensusNumber},
\code{Family}, \code{Tag}, and \code{HOM}. \code{ExactDate} should contain dates from
1980-01-01 to the present day in the format yyyy-mm-dd.

\item[\code{...}] Expressions to pick main stems of a specific \code{dbh} range (e.g.
\code{DBH >= 10} or \code{DBH >= 10, DBH < 20}, or \code{DBH >= 10 \& DBH < 20}).
\end{ldescription}
\end{Arguments}
%
\begin{Details}\relax
You don't need to pick stems by status before feeding data to these
functions. Doing so may make your code more readable but it should not affect
the result. This is because the expressions passed to \code{...} pick data by
\code{dbh} and exclude the missing \code{dbh} values associated to non-alive stems,
including dead, missing, and gone stems.
\end{Details}
%
\begin{Value}
A dataframe.
\end{Value}
%
\begin{SeeAlso}\relax
\code{\LinkA{fgeo.tool::pick\_main\_stem()}{fgeo.tool::pick.Rul.main.Rul.stem()}}.

Other functions for abundance and basal area: \code{\LinkA{abundance}{abundance}}
\end{SeeAlso}
%
\begin{Examples}
\begin{ExampleCode}
library(fgeo.tool)

# Example data
vft <- tibble(
  PlotName = c("luq", "luq", "luq", "luq", "luq", "luq", "luq", "luq"),
  CensusID = c(1L, 1L, 1L, 1L, 2L, 2L, 2L, 2L),
  TreeID = c(1L, 1L, 2L, 2L, 1L, 1L, 2L, 2L),
  StemID = c(1.1, 1.2, 2.1, 2.2, 1.1, 1.2, 2.1, 2.2),
  Status = c(
    "alive", "dead", "alive", "alive", "alive", "gone",
    "dead", "dead"
  ),
  DBH = c(10L, NA, 20L, 30L, 20L, NA, NA, NA),
  Genus = c("Gn", "Gn", "Gn", "Gn", "Gn", "Gn", "Gn", "Gn"),
  SpeciesName = c("spp", "spp", "spp", "spp", "spp", "spp", "spp", "spp"),
  ExactDate = c(
    "2001-01-01", "2001-01-01", "2001-01-01", "2001-01-01",
    "2002-01-01", "2002-01-01", "2002-01-01",
    "2002-01-01"
  ),
  PlotCensusNumber = c(1L, 1L, 1L, 1L, 2L, 2L, 2L, 2L),
  Family = c("f", "f", "f", "f", "f", "f", "f", "f"),
  Tag = c(1L, 1L, 2L, 2L, 1L, 1L, 2L, 2L),
  HOM = c(130L, 130L, 130L, 130L, 130L, 130L, 130L, 130L)
)

vft

abundance_byyr(vft, DBH >= 10, DBH < 20)

abundance_byyr(vft, DBH >= 10)

basal <- basal_area_byyr(vft, DBH >= 10)
basal
## Not run: 
measurements_is_installed <- requireNamespace("measurements", quietly = TRUE)
if (measurements_is_installed) {
  # Convert units
  years <- c("yr_2001", "yr_2002")
  basal_he <- basal %>%
    purrr::modify_at(
      years,
      ~ measurements::conv_unit(.x, from = "mm2", to = "hectare")
    )
  basal_he

  # Standardize
  number_of_hectares <- 50
  basal_he %>%
    purrr::map_at(years, ~ .x / number_of_hectares)
}

## End(Not run)
\end{ExampleCode}
\end{Examples}
\inputencoding{utf8}
\HeaderA{fgeo\_habitat}{Create habitat data from measures of topography.}{fgeo.Rul.habitat}
%
\begin{Description}\relax
This function constructs habitat data based on elevation data. It calculates
habitats in two steps:
\begin{enumerate}

\item It calculates mean elevation, convexity and slope for each quadrat.
\item It calculates habitats based on hierarchical clustering of the topographic
metrics from step 1.

\end{enumerate}

\end{Description}
%
\begin{Usage}
\begin{verbatim}
fgeo_habitat(elev, gridsize, n, ...)
\end{verbatim}
\end{Usage}
%
\begin{Arguments}
\begin{ldescription}
\item[\code{elev}] One of these:
\begin{itemize}

\item A dataframe containing elevation data, with columns \code{gx}, \code{gy}, and
\code{elev}, or \code{x}, \code{y}, and \code{elev} (e.g. \code{fgeo.x::elevation\$col}).
\item A ForestGEO-like elevation list with elements \code{xdim} and \code{ydim} giving
plot dimensions, and element \code{col} containing a dataframe as described in
the previous item (e.g. \code{fgeo.x::elevation}).

\end{itemize}


\item[\code{gridsize}] Number giving the size of each quadrat for which a habitat
is calculated. Commonly, \code{gridsize = 20}.

\item[\code{n}] Integer. Number of cluster-groups to construct (passed to the
argument \code{k} to \code{\LinkA{stats::cutree()}{stats::cutree()}}).

\item[\code{...}] Arguments passed to \code{\LinkA{fgeo\_topography()}{fgeo.Rul.topography}}.
\end{ldescription}
\end{Arguments}
%
\begin{Value}
A dataframe of subclass fgeo\_habitat, with columns \code{gx} and \code{gy},
rounded with accuracy determined by \code{gridsize}, and column \code{habitats}, with
as many distinct integer values as determined by the argument \code{n}.
\end{Value}
%
\begin{Author}\relax
Rick Condit.
\end{Author}
%
\begin{SeeAlso}\relax
\code{fgeo.plot::autoplot.fgeo\_habitat()}, \code{\LinkA{fgeo\_topography()}{fgeo.Rul.topography}}.

Other habitat functions: \code{\LinkA{fgeo\_topography}{fgeo.Rul.topography}},
\code{\LinkA{tt\_test}{tt.Rul.test}}

Other functions to construct fgeo classes: \code{\LinkA{fgeo\_topography}{fgeo.Rul.topography}}
\end{SeeAlso}
%
\begin{Examples}
\begin{ExampleCode}
assert_is_installed("fgeo.x")

# Input a ForestGEO-like elevation list or dataframe
elevation_ls <- fgeo.x::elevation
habitats <- fgeo_habitat(
  elevation_ls,
  gridsize = 20, n = 4
)

str(habitats)

## Not run: 
assert_is_installed("fgeo.plot")
library(fgeo.plot)

autoplot(habitats)

## End(Not run)

# Habitat data is useful for calculating species-habitat associations
census <- fgeo.x::tree6_3species
as_tibble(
  tt_test(census, habitats)
)
\end{ExampleCode}
\end{Examples}
\inputencoding{utf8}
\HeaderA{fgeo\_topography}{Create topography data: convexity, slope, and mean elevation.}{fgeo.Rul.topography}
\methaliasA{fgeo\_topography.data.frame}{fgeo\_topography}{fgeo.Rul.topography.data.frame}
\methaliasA{fgeo\_topography.list}{fgeo\_topography}{fgeo.Rul.topography.list}
%
\begin{Description}\relax
Create topography data: convexity, slope, and mean elevation.
\end{Description}
%
\begin{Usage}
\begin{verbatim}
fgeo_topography(elev, ...)

## S3 method for class 'data.frame'
fgeo_topography(elev, gridsize, xdim = NULL,
  ydim = NULL, edgecorrect = TRUE, ...)

## S3 method for class 'list'
fgeo_topography(elev, gridsize, edgecorrect = TRUE, ...)
\end{verbatim}
\end{Usage}
%
\begin{Arguments}
\begin{ldescription}
\item[\code{elev}] One of these:
\begin{itemize}

\item A dataframe containing elevation data, with columns \code{gx}, \code{gy}, and
\code{elev}, or \code{x}, \code{y}, and \code{elev} (e.g. \code{fgeo.x::elevation\$col}).
\item A ForestGEO-like elevation list with elements \code{xdim} and \code{ydim} giving
plot dimensions, and element \code{col} containing a dataframe as described in
the previous item (e.g. \code{fgeo.x::elevation}).

\end{itemize}


\item[\code{...}] Other arguments passed to methods.

\item[\code{gridsize}] Number giving the size of each quadrat for which a habitat
is calculated. Commonly, \code{gridsize = 20}.

\item[\code{xdim, ydim}] (Required if \code{elev} is a dataframe) \code{x} and \code{y}
dimensions of the plot.

\item[\code{edgecorrect}] Correct convexity in edge quadrats?
\end{ldescription}
\end{Arguments}
%
\begin{Value}
A dataframe of subclass fgeo\_topography.
\end{Value}
%
\begin{Section}{Acknowledgment}

Thanks to Jian Zhang for reporting a bug (issue 59).
\end{Section}
%
\begin{Author}\relax
This function wraps code by Rick Condit.
\end{Author}
%
\begin{SeeAlso}\relax
\code{\LinkA{fgeo\_habitat()}{fgeo.Rul.habitat}}.

Other habitat functions: \code{\LinkA{fgeo\_habitat}{fgeo.Rul.habitat}},
\code{\LinkA{tt\_test}{tt.Rul.test}}

Other functions to construct fgeo classes: \code{\LinkA{fgeo\_habitat}{fgeo.Rul.habitat}}
\end{SeeAlso}
%
\begin{Examples}
\begin{ExampleCode}
assert_is_installed("fgeo.x")

elev_list <- fgeo.x::elevation
fgeo_topography(elev_list, gridsize = 20)

elev_df <- elev_list$col
fgeo_topography(elev_df, gridsize = 20, xdim = 320, ydim = 500)
\end{ExampleCode}
\end{Examples}
\inputencoding{utf8}
\HeaderA{recruitment\_ctfs}{Recruitment, mortality, and growth.}{recruitment.Rul.ctfs}
\aliasA{growth\_ctfs}{recruitment\_ctfs}{growth.Rul.ctfs}
\aliasA{mortality\_ctfs}{recruitment\_ctfs}{mortality.Rul.ctfs}
%
\begin{Description}\relax
These functions are adapted from the CTFS-R package. Compared to the
original functions, these ones have a similar interface but use more
conservative defaults and allow suppressing messages. These functions also
feature formal tests, bug fixes, additional assertions, and improved
messages.
\end{Description}
%
\begin{Usage}
\begin{verbatim}
recruitment_ctfs(census1, census2, mindbh = NULL, alivecode = NULL,
  split1 = NULL, split2 = NULL, quiet = FALSE)

mortality_ctfs(census1, census2, alivecode = NULL, split1 = NULL,
  split2 = NULL, quiet = FALSE)

growth_ctfs(census1, census2, rounddown = FALSE, method = "I",
  stdev = FALSE, dbhunit = "mm", mindbh = NULL, growthcol = "dbh",
  err.limit = 1000, maxgrow = 1000, split1 = NULL, split2 = NULL,
  quiet = FALSE)
\end{verbatim}
\end{Usage}
%
\begin{Arguments}
\begin{ldescription}
\item[\code{census1, census2}] Two census tables, each being a ForestGEO-like \emph{tree}
table (dataframe). A \emph{stem} table won't fail, but you should use a \emph{tree}
table because demography analyses make more sense at the scale of trees
than at the scale of stems.

\item[\code{mindbh}] The minimum diameter above which the counts are done. Trees
smaller than \code{mindbh} are excluded. By default all living trees of any size
are included.

\item[\code{alivecode}] Character; valid values of \code{status} indicating that a tree
is alive. The default, 'A', is the standard CTFS designation for living
trees or stems.

\item[\code{split1, split2}] Optional vector(s) to aggregate results by. Each vector
should be a column of either \code{census1} or \code{census2}. The default aggregates
the result across the entire census datasets.

\item[\code{quiet}] Use \code{TRUE} to suppress messages.

\item[\code{rounddown}] If \code{TRUE}, all \code{dbh < 55} are rounded down to the nearest
multiple of 5.

\item[\code{method}] Either "I" or "E":
\begin{itemize}

\item Use "I" to calculate annual dbh increment as \code{(dbh2 - dbh1)/time}
\item Use "E" to calculate the relative growth rate as
\code{(log(dbh2) - log(dbh1)) / time}

\end{itemize}


\item[\code{stdev}] Logical:
\begin{itemize}

\item \code{FALSE} returns confidence limits.
\item \code{TRUE} returns the SD in growth rate per group.

\end{itemize}


\item[\code{dbhunit}] "cm" or "mm".

\item[\code{growthcol}] Either "dbh" or "agb" to define how growth is measured.

\item[\code{err.limit, maxgrow}] A number. Numbers such as 10000 are high and will
return all measures.
\end{ldescription}
\end{Arguments}
%
\begin{Details}\relax
Survivors are all individuals alive in both censuses, with \code{status == A} in
the first census, and a diameter greater than \code{mindbh} in the first census.
The total population in the second census includes all those alive plus any
other survivors. Individuals whose status is \code{NA} in either census are
deleted from all calculations.
\end{Details}
%
\begin{Value}
Metrics of recruitment: Similar to metrics of mortality.

Metrics of mortality:
\begin{itemize}

\item \code{N}: the number of individuals alive in the census 1 per category
selected.
\item \code{D}: the number of individuals no longer alive in census 2.
\item \code{rate}: the mean annualized mortality rate constant per category
selected, calculated as (log(N)-log(S))/time.
\item \code{upper}: upper confidence limit of mean rate.
\item \code{lower}: lower confidence limit of mean rate.
\item \code{time}: mean time interval in years.
\item \code{date1}: mean date included individuals were measured in census 1, as
julian object (R displays as date, but treats as integer).
\item \code{date2}: mean date in census 2.
\item \code{dbhmean}: mean dbh in census 1 of individuals included.

\end{itemize}


Metrics of growth:
\begin{itemize}

\item \code{rate}, the mean annualized growth rate per category selected, either dbh
increment, or relative growth.
\item \code{N}, the number of individuals included in the mean (not counting any
excluded).
\item \code{clim} (or sd with \code{stdev = TRUE}), width of confidence interval; add this
number to the mean rate to get upper confidence limit, substract to get
lower.
\item \code{dbhmean}, mean dbh in census 1 of individuals included.
\item \code{time}, mean time interval in years.
\item \code{date1}, mean date included individuals were measured in census 1, as
julian object (R displays as date, but treats as integer).
\item \code{date2}, mean date in census 2.

\end{itemize}

\end{Value}
%
\begin{Author}\relax
Rick Condit, Suzanne Lao.
\end{Author}
%
\begin{Examples}
\begin{ExampleCode}
assert_is_installed("fgeo.x")

census1 <- fgeo.x::tree5
census2 <- fgeo.x::tree6

as_tibble(
  recruitment_ctfs(census1, census2)
)

# Use `interaction(...)` to aggregate by any number of grouping variables
sp_quadrat <- interaction(census1$sp, census1$quadrat)

recruitment <- recruitment_ctfs(
  census1, census2,
  split1 = sp_quadrat,
  quiet = TRUE
)
as_tibble(recruitment)

mortality <- mortality_ctfs(
  census1, census2,
  split1 = sp_quadrat, quiet = TRUE
)
as_tibble(mortality)

growth <- growth_ctfs(census1, census2, split1 = sp_quadrat, quiet = TRUE)
as_tibble(growth)

# Easy way to separate grouping variables
tidyr_is_installed <- requireNamespace("tidyr", quietly = TRUE)
if (tidyr_is_installed) {
  library(tidyr)

  as_tibble(growth) %>%
    separate(groups, into = c("sp", "quadrat"))
}
\end{ExampleCode}
\end{Examples}
\inputencoding{utf8}
\HeaderA{summary.tt\_df}{Summary of \code{tt\_test()} results.}{summary.tt.Rul.df}
\aliasA{summary.tt\_lst}{summary.tt\_df}{summary.tt.Rul.lst}
%
\begin{Description}\relax
Summary of \code{tt\_test()} results.
\end{Description}
%
\begin{Usage}
\begin{verbatim}
## S3 method for class 'tt_df'
summary(object, ...)

## S3 method for class 'tt_lst'
summary(object, ...)
\end{verbatim}
\end{Usage}
%
\begin{Arguments}
\begin{ldescription}
\item[\code{object}] An object of class "tt\_df" or "tt\_lst".

\item[\code{...}] Not used (included only for compatibility with \code{summary}).
\end{ldescription}
\end{Arguments}
%
\begin{Value}
A tibble.
\end{Value}
%
\begin{Author}\relax
Adapted from code contributed by Daniel Zuleta.
\end{Author}
%
\begin{SeeAlso}\relax
\code{\LinkA{tt\_test()}{tt.Rul.test}}, \code{\LinkA{base::summary()}{base::summary()}}.
\end{SeeAlso}
%
\begin{Examples}
\begin{ExampleCode}
assert_is_installed("fgeo.x")

tt_result <- tt_test(fgeo.x::tree6_3species, fgeo.x::habitat)

summary(tt_result)

# Same
summary(as_tibble(tt_result))

# You may want to add the explanation to the result of `tt_test()`

dplyr::left_join(as_tibble(tt_result), summary(tt_result))

# You may prefer a wide matrix
Reduce(rbind, tt_result)

# You may prefer a wide dataframe
tidyr::spread(summary(tt_result), "habitat", "association")
\end{ExampleCode}
\end{Examples}
\inputencoding{utf8}
\HeaderA{tt\_test}{Torus Translation Test to determine habitat associations of tree species.}{tt.Rul.test}
%
\begin{Description}\relax
Determine habitat-species associations with code developed by Sabrina Russo,
Daniel Zuleta, Matteo Detto, and Kyle Harms.
\end{Description}
%
\begin{Usage}
\begin{verbatim}
tt_test(tree, habitat, sp = NULL, plotdim = NULL, gridsize = NULL)
\end{verbatim}
\end{Usage}
%
\begin{Arguments}
\begin{ldescription}
\item[\code{tree}] A dataframe; a ForestGEO \emph{tree} table (see details).

\item[\code{habitat}] Object giving the habitat designation for each
plot partition defined by \code{gridsize}. See \code{\LinkA{fgeo\_habitat()}{fgeo.Rul.habitat}}.

\item[\code{sp}] Character sting giving any number of species-names.

\item[\code{plotdim, gridsize}] Plot dimensions and gridsize. If \code{NULL} (default)
they will be guessed, and a message will inform you of the chosen values.
If the guess is wrong, you should provide the correct values manually (and
check that your habitat data is correct).
\end{ldescription}
\end{Arguments}
%
\begin{Details}\relax
This test only makes sense at the population level. We are interested in
knowing whether or not individuals of a species are aggregated on a habitat.
Multiple stems of an individual do not represent population level processes
but individual level processes. Thus, you should use data of individual trees
-- i.e. use a \emph{tree} table, and not a \emph{stem} table with potentially multiple
stems per tree.

You should only try to determine the habitat association for sufficiently
abundant species. In a 50-ha plot, a minimum abundance of 50 trees/species
has been used.
\end{Details}
%
\begin{Value}
A list of matrices.
\end{Value}
%
\begin{Section}{Acknowledgments}

Nestor Engone Obiang, David Kenfack, Jennifer Baltzer, and Rutuja
Chitra-Tarak provided feedback. Daniel Zuleta provided guidance.
\end{Section}
%
\begin{Section}{Interpretation of Output}

\begin{itemize}

\item \code{N.Hab.1}: Count of stems of the focal species in habitat 1.
\item \code{Gr.Hab.1}: Count of instances the observed relative density of the focal
species on habitat 1 was greater than the relative density based on the TT
habitat map.
\item \code{Ls.Hab.1}: Count of instances the observed relative density of the focal
species on habitat 1 was less than the relative density based on the TT
habitat map.
\item \code{Eq.Hab.1}: Count of instances the observed relative density of the focal
species on habitat 1 was equal to the relative density based on the TT
habitat map.
The sum of the \code{Gr.Hab.x}, \code{Ls.Hab.x}, and \code{Eq.Hab.x} columns for one habitat
equals the number of 20 x20 quads in the plot.
The \code{Rep.Agg.Neut} columns for each habitat indicate whether the species is
significantly repelled (-1), aggregated (1), or neutrally distributed (0) on
the habitat in question.

\end{itemize}


The probabilities associated with the test for whether these patterns are
statistically significant are in the \code{Obs.Quantile} columns for each habitat.
Note that to calculate the probability for \emph{repelled}, it is \emph{the value
given}, but to calculate the probability for \emph{aggregated}, it is \emph{one minus
the value given}.

Values of the \code{Obs.Quantile < 0.025} means that the species is repelled from
that habitat, while values of the \code{Obs.Quantile > 0.975} means that the
species is aggregated on that habitat.
\end{Section}
%
\begin{Section}{References}

Zuleta, D., Russo, S.E., Barona, A. et al. Plant Soil (2018).
\url{https://doi.org/10.1007/s11104-018-3878-0}.
\end{Section}
%
\begin{Author}\relax
Sabrina Russo, Daniel Zuleta, Matteo Detto, and Kyle Harms.
\end{Author}
%
\begin{SeeAlso}\relax
\code{\LinkA{summary.tt\_lst()}{summary.tt.Rul.lst}}, \code{\LinkA{summary.tt\_df()}{summary.tt.Rul.df}}, \code{\LinkA{as\_tibble()}{as.Rul.tibble}},
\code{\LinkA{fgeo\_habitat()}{fgeo.Rul.habitat}}.

Other habitat functions: \code{\LinkA{fgeo\_habitat}{fgeo.Rul.habitat}},
\code{\LinkA{fgeo\_topography}{fgeo.Rul.topography}}
\end{SeeAlso}
%
\begin{Examples}
\begin{ExampleCode}
library(fgeo.tool)
assert_is_installed("fgeo.x")

# Example data
tree <- fgeo.x::tree6_3species
elevation <- fgeo.x::elevation

# Pick alive trees, of 10 mm or more
census <- filter(tree, status == "A", dbh >= 10)

# Pick sufficiently abundant species
pick <- filter(dplyr::add_count(census, sp), n > 50)

# Use your habitat data or create it from elevation data
habitat <- fgeo_habitat(elevation, gridsize = 20, n = 4)

# Defaults to using all species
as_tibble(
  tt_test(census, habitat)
)

Reduce(rbind, tt_test(census, habitat))

some_species <- c("CASARB", "PREMON")
result <- tt_test(census, habitat, sp = some_species)
summary(result)
\end{ExampleCode}
\end{Examples}
\printindex{}
\end{document}
