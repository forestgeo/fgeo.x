\nonstopmode{}
\documentclass[a4paper]{book}
\usepackage[times,inconsolata,hyper]{Rd}
\usepackage{makeidx}
\usepackage[utf8]{inputenc} % @SET ENCODING@
% \usepackage{graphicx} % @USE GRAPHICX@
\makeindex{}
\begin{document}
\chapter*{}
\begin{center}
{\textbf{\huge Package `fgeo'}}
\par\bigskip{\large \today}
\end{center}
\begin{description}
\raggedright{}
\inputencoding{utf8}
\item[Title]\AsIs{Analyze Forest Diversity and Dynamics}
\item[Version]\AsIs{1.1.1}
\item[Description]\AsIs{Install, load, and access the documentation of
multiple packages to analyze forest diversity and dynamics. 'fgeo'
allows you to manipulate and plot ForestGEO data, and to do common
analyses including abundance, demography, and species-habitats
associations (<http://www.forestgeo.si.edu/>).}
\item[License]\AsIs{GPL-3}
\item[URL]\AsIs{}\url{http://forestgeo.github.io/fgeo}\AsIs{, }\url{https://github.com/forestgeo/fgeo}\AsIs{}
\item[BugReports]\AsIs{}\url{https://github.com/forestgeo/fgeo/issues}\AsIs{}
\item[Imports]\AsIs{cli (>= 1.0.1), crayon (>= 1.3.4), dplyr (>= 0.7.8),
fgeo.analyze (>= 1.1.2), fgeo.plot (>= 1.1.1), fgeo.tool (>=
1.2.1), fgeo.x (>= 1.1.0), glue (>= 1.3.0), magrittr (>= 1.5),
purrr (>= 0.3.0), rlang (>= 0.3.1), rstudioapi (>= 0.9.0),
utils}
\item[Suggests]\AsIs{covr (>= 3.2.1), DT (>= 0.5), knitr (>= 1.21), rmarkdown (>=
1.11), spelling (>= 2.0), testthat (>= 2.0.1)}
\item[Additional\_repositories]\AsIs{https://forestgeo.github.io/drat/}
\item[Encoding]\AsIs{UTF-8}
\item[Language]\AsIs{en-US}
\item[LazyData]\AsIs{true}
\item[Roxygen]\AsIs{list(markdown = TRUE)}
\item[RoxygenNote]\AsIs{6.1.1}
\item[NeedsCompilation]\AsIs{no}
\item[Author]\AsIs{Mauro Lepore [aut, ctr, cre],
Gabriel Arellano [aut, rev],
Richard Condit [aut],
Stuart Davies [aut, rev],
Matteo Detto [aut],
Pamela Hall [aut],
Kyle Harms [aut],
David Kenfack [aut, rev],
Lauren Krizel [rev],
Suzanne Lao [aut, rev],
Sean McMahon [aut, rev],
Haley Overstreet [rev],
Sabrina Russo [aut, rev],
Kristina Teixeira [aut, rev],
Graham Zemunik [aut, rev],
Daniel Zuleta [aut, rev],
CTFS-ForestGEO [cph, fnd]}
\item[Maintainer]\AsIs{Mauro Lepore }\email{leporem@si.edu}\AsIs{}
\end{description}
\Rdcontents{\R{} topics documented:}
\inputencoding{utf8}
\HeaderA{fgeo\_browse}{Open a web browser on fgeo's website.}{fgeo.Rul.browse}
\aliasA{fgeo\_browse\_reference}{fgeo\_browse}{fgeo.Rul.browse.Rul.reference}
%
\begin{Description}\relax
Load fgeo's URLs into an HTML browser.
\end{Description}
%
\begin{Usage}
\begin{verbatim}
fgeo_browse()

fgeo_browse_reference()
\end{verbatim}
\end{Usage}
%
\begin{SeeAlso}\relax
Other functions to explore fgeo: \code{\LinkA{fgeo\_help}{fgeo.Rul.help}}
\end{SeeAlso}
%
\begin{Examples}
\begin{ExampleCode}
if (interactive()) {
  fgeo_browse()
  fgeo_browse_reference()
}
\end{ExampleCode}
\end{Examples}
\inputencoding{utf8}
\HeaderA{fgeo\_help}{Get help with fgeo.}{fgeo.Rul.help}
%
\begin{Description}\relax
\code{\LinkA{fgeo\_help()}{fgeo.Rul.help}} finds all \strong{fgeo} help files. You can refine
your search directly on the viewer panel of RStudio or via a string passed
as the first argument to \code{\LinkA{fgeo\_help()}{fgeo.Rul.help}}.
\end{Description}
%
\begin{Usage}
\begin{verbatim}
fgeo_help(pattern = "", package = NULL, rebuild = TRUE, ...)
\end{verbatim}
\end{Usage}
%
\begin{Arguments}
\begin{ldescription}
\item[\code{pattern}] A character string to be matched in the name, alias or title
of a topic's documentation. Defaults to matching everything, which
retrieves all the documentation of \strong{fgeo} packages.

\item[\code{package}] A character string giving the name of one or more
packages to limit the search, or \code{NULL} to search all fgeo packages.

\item[\code{rebuild}] a logical indicating whether the help database should
be rebuilt.  This will be done automatically if \code{lib.loc} or
the search path is changed, or if \code{package} is used and a value
is not found.

\item[\code{...}] Other arguments passed to \code{\LinkA{utils::help.search()}{utils::help.search()}}.
\end{ldescription}
\end{Arguments}
%
\begin{Value}
The results are returned in a list object of class "hsearch", which
has a print method for nicely formatting the results of the query.
\end{Value}
%
\begin{SeeAlso}\relax
\code{\LinkA{utils::help.search()}{utils::help.search()}}.

Other functions to explore fgeo: \code{\LinkA{fgeo\_browse}{fgeo.Rul.browse}}
\end{SeeAlso}
%
\begin{Examples}
\begin{ExampleCode}
if (interactive()) {
  fgeo_help()
}

dplyr::as_tibble(fgeo_help()$matches)

if (interactive()) {
  fgeo_help("stem", package = "fgeo.x")
}
\end{ExampleCode}
\end{Examples}
\printindex{}
\end{document}
