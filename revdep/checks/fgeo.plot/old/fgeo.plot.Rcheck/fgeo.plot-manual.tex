\nonstopmode{}
\documentclass[a4paper]{book}
\usepackage[times,inconsolata,hyper]{Rd}
\usepackage{makeidx}
\usepackage[utf8]{inputenc} % @SET ENCODING@
% \usepackage{graphicx} % @USE GRAPHICX@
\makeindex{}
\begin{document}
\chapter*{}
\begin{center}
{\textbf{\huge Package `fgeo.plot'}}
\par\bigskip{\large \today}
\end{center}
\begin{description}
\raggedright{}
\inputencoding{utf8}
\item[Title]\AsIs{Plot ForestGEO Data}
\item[Version]\AsIs{1.1.1}
\item[Description]\AsIs{Plot ForestGEO data
(<http://www.forestgeo.si.edu/>).}
\item[License]\AsIs{GPL-3}
\item[URL]\AsIs{}\url{https://github.com/forestgeo/fgeo.plot}\AsIs{,
}\url{https://forestgeo.github.io/fgeo.plot/}\AsIs{}
\item[BugReports]\AsIs{}\url{https://github.com/forestgeo/fgeo.plot/issues}\AsIs{}
\item[Depends]\AsIs{R (>= 3.3)}
\item[Imports]\AsIs{dplyr (>= 0.7.8), fgeo.tool (>= 1.2.0), ggplot2 (>= 3.1.0),
ggrepel (>= 0.8.0), glue (>= 1.3.0), magrittr (>= 1.5), purrr
(>= 0.3.0), rlang (>= 0.3.1), stats, stringr (>= 1.3.1)}
\item[Suggests]\AsIs{covr (>= 3.2.1), fgeo.x (>= 1.1.0), gridExtra (>= 2.3), knitr
(>= 1.21), rmarkdown (>= 1.11), spelling (>= 2.0), testthat (>=
2.0.1)}
\item[Additional\_repositories]\AsIs{https://forestgeo.github.io/drat/}
\item[Encoding]\AsIs{UTF-8}
\item[Language]\AsIs{en-US}
\item[LazyData]\AsIs{true}
\item[Roxygen]\AsIs{list(markdown = TRUE)}
\item[RoxygenNote]\AsIs{6.1.1}
\item[NeedsCompilation]\AsIs{no}
\item[Author]\AsIs{Mauro Lepore [aut, ctr, cre],
CTFS-ForestGEO [cph, fnd]}
\item[Maintainer]\AsIs{Mauro Lepore }\email{leporem@si.edu}\AsIs{}
\end{description}
\Rdcontents{\R{} topics documented:}
\inputencoding{utf8}
\HeaderA{autoplot.fgeo\_habitat}{Plot habitats.}{autoplot.fgeo.Rul.habitat}
%
\begin{Description}\relax
Plot habitats.
\end{Description}
%
\begin{Usage}
\begin{verbatim}
## S3 method for class 'fgeo_habitat'
autoplot(object, ...)
\end{verbatim}
\end{Usage}
%
\begin{Arguments}
\begin{ldescription}
\item[\code{object}] An object of class "fgeo\_habitat" (see \code{fgeo\_habitat} at
\url{http://bit.ly/fgeo-reference}).

\item[\code{...}] Not used (included for compatibility across methods).
\end{ldescription}
\end{Arguments}
%
\begin{Value}
An object of class "ggplot".
\end{Value}
%
\begin{SeeAlso}\relax
Other plot functions: \code{\LinkA{autoplot.sp\_elev}{autoplot.sp.Rul.elev}},
\code{\LinkA{autoplot\_by\_species.sp\_elev}{autoplot.Rul.by.Rul.species.sp.Rul.elev}},
\code{\LinkA{elev}{elev}},
\code{\LinkA{plot\_dbh\_bubbles\_by\_quadrat}{plot.Rul.dbh.Rul.bubbles.Rul.by.Rul.quadrat}},
\code{\LinkA{plot\_tag\_status\_by\_subquadrat}{plot.Rul.tag.Rul.status.Rul.by.Rul.subquadrat}},
\code{\LinkA{sp\_elev}{sp.Rul.elev}}, \code{\LinkA{sp}{sp}}

Other autoplots: \code{\LinkA{autoplot.sp\_elev}{autoplot.sp.Rul.elev}},
\code{\LinkA{elev}{elev}}, \code{\LinkA{sp\_elev}{sp.Rul.elev}},
\code{\LinkA{sp}{sp}}
\end{SeeAlso}
%
\begin{Examples}
\begin{ExampleCode}
## Not run: 
assert_is_installed("fgeo.x")
assert_is_installed("fgeo.analyze")
library(fgeo.analyze)

habitats <- fgeo_habitat(fgeo.x::elevation, gridsize = 20, n = 4)
autoplot(habitats)

## End(Not run)
\end{ExampleCode}
\end{Examples}
\inputencoding{utf8}
\HeaderA{autoplot.sp\_elev}{Plot species distribution and/or topography.}{autoplot.sp.Rul.elev}
\aliasA{autoplot.elev}{autoplot.sp\_elev}{autoplot.elev}
\aliasA{autoplot.sp}{autoplot.sp\_elev}{autoplot.sp}
%
\begin{Description}\relax
Plot the columns \code{sp} and/or \code{elev} of ForestGEO-like datasets of class 'sp'
and/or 'sp\_elev'.
\begin{itemize}

\item You can create a 'sp' \code{object} with:

\end{itemize}


\begin{alltt}    object <- sp(DATA-WITH-VARIABLE-sp)
\end{alltt}

\begin{itemize}

\item You can create an 'elev' \code{object} with:

\end{itemize}


\begin{alltt}    object <- elev(DATA-WITH-VARIABLE-elev)
\end{alltt}


\begin{itemize}

\item You can create a 'sp\_elev' \code{object} with:

\end{itemize}


\begin{alltt}    object <- sp_elev(DATA-WITH-VARIABLE-sp, DATA-WITH-VARIABLE-elev)
\end{alltt}


See \strong{Examples} below.
\end{Description}
%
\begin{Usage}
\begin{verbatim}
## S3 method for class 'sp_elev'
autoplot(object, fill = "sp",
  hide_fill_legend = FALSE, shape = 21, point_size = 3,
  facet = TRUE, contour_size = 0.5, low = "blue", high = "red",
  hide_color_legend = FALSE, bins = NULL,
  add_elevation_labels = TRUE, label_size = 3, label_color = "grey",
  xyjust = 1, fontface = "italic", xlim = NULL, ylim = NULL,
  custom_theme = NULL, ...)

## S3 method for class 'sp'
autoplot(object, fill = "sp", hide_fill_legend = FALSE,
  shape = 21, point_size = 3, facet = TRUE, xlim = NULL,
  ylim = NULL, custom_theme = NULL, ...)

## S3 method for class 'elev'
autoplot(object, contour_size = 0.5, low = "blue",
  high = "red", hide_color_legend = FALSE, bins = NULL,
  add_elevation_labels = TRUE, label_size = 3, label_color = "grey",
  xyjust = 1, fontface = "italic", xlim = NULL, ylim = NULL,
  custom_theme = NULL, ...)
\end{verbatim}
\end{Usage}
%
\begin{Arguments}
\begin{ldescription}
\item[\code{object}] An object created with \code{\LinkA{sp()}{sp}}, \code{\LinkA{elev()}{elev}}, or \code{\LinkA{sp\_elev()}{sp.Rul.elev}}.

\item[\code{fill}] Character; either a color or "sp", which maps each species to a
different color.

\item[\code{hide\_fill\_legend}] Logical; \code{TRUE} hides the fill legend.

\item[\code{shape}] A number giving point shape (as in \code{\LinkA{graphics::points()}{graphics::points()}}). Passed
to \code{\LinkA{ggplot2::geom\_point()}{ggplot2::geom.Rul.point()}}.

\item[\code{point\_size}] A number giving point size. Passed to
\code{\LinkA{ggplot2::geom\_point()}{ggplot2::geom.Rul.point()}}.

\item[\code{facet}] Logical; \code{TRUE} wraps multiple panels within the area of a single
graphic plot.

\item[\code{contour\_size}] A number giving the size of the contour of elevation
lines. Passed to \code{ggplot2::stat\_contour()} (see \code{\LinkA{ggplot2::geom\_contour()}{ggplot2::geom.Rul.contour()}}).

\item[\code{low, high}] A string giving a color of the elevation lines representing
low and high elevation.

\item[\code{hide\_color\_legend}] Logical; \code{TRUE} hides the color legend.

\item[\code{bins}] A number giving the number of elevation lines to plot.

\item[\code{add\_elevation\_labels}] Logical; \code{FALSE} hides elevation labels.

\item[\code{label\_size, label\_color, fontface}] A number (\code{label\_size}) or character
string (\code{label\_color} and \code{fontface}) giving the size, colour and fontface
of the text labels for the elevation lines.

\item[\code{xyjust}] A number to adjust the position of the text labels of the
elevation lines.

\item[\code{xlim, ylim}] A vector of length 2, for example \code{c(0, 500)}, giving the
minimum and maximum limits of the vertical and horizontal coordinates.

\item[\code{custom\_theme}] A valid \code{\LinkA{ggplot2::theme()}{ggplot2::theme()}}. \code{NULL} uses the default
theme \code{\LinkA{theme\_default()}{theme.Rul.default}}.

\item[\code{...}] Not used (included for compatibility across methods).
\end{ldescription}
\end{Arguments}
%
\begin{Details}\relax
\code{autoplot(sp\_elev(DATA-WITH-VARIABLE-sp)} (without elevation data) is
equivalent to \code{autoplot(sp(DATA-WITH-VARIABLE-sp))}.

\strong{fgeo.plot} wraps some functions from the \strong{ggplot2} package.
For more control you can use \strong{ggplot2} directly.
\end{Details}
%
\begin{Value}
A "ggplot".
\end{Value}
%
\begin{SeeAlso}\relax
Other plot functions: \code{\LinkA{autoplot.fgeo\_habitat}{autoplot.fgeo.Rul.habitat}},
\code{\LinkA{autoplot\_by\_species.sp\_elev}{autoplot.Rul.by.Rul.species.sp.Rul.elev}},
\code{\LinkA{elev}{elev}},
\code{\LinkA{plot\_dbh\_bubbles\_by\_quadrat}{plot.Rul.dbh.Rul.bubbles.Rul.by.Rul.quadrat}},
\code{\LinkA{plot\_tag\_status\_by\_subquadrat}{plot.Rul.tag.Rul.status.Rul.by.Rul.subquadrat}},
\code{\LinkA{sp\_elev}{sp.Rul.elev}}, \code{\LinkA{sp}{sp}}

Other autoplots: \code{\LinkA{autoplot.fgeo\_habitat}{autoplot.fgeo.Rul.habitat}},
\code{\LinkA{elev}{elev}}, \code{\LinkA{sp\_elev}{sp.Rul.elev}},
\code{\LinkA{sp}{sp}}

Other functions to plot elevation: \code{\LinkA{autoplot\_by\_species.sp\_elev}{autoplot.Rul.by.Rul.species.sp.Rul.elev}},
\code{\LinkA{elev}{elev}}, \code{\LinkA{sp\_elev}{sp.Rul.elev}}

Other functions to plot species: \code{\LinkA{autoplot\_by\_species.sp\_elev}{autoplot.Rul.by.Rul.species.sp.Rul.elev}},
\code{\LinkA{sp\_elev}{sp.Rul.elev}}, \code{\LinkA{sp}{sp}}
\end{SeeAlso}
%
\begin{Examples}
\begin{ExampleCode}
assert_is_installed("fgeo.x")

# Species ---------------------------------------------------------------

# Small dataset with a few species for quick examples
census <- fgeo.x::download_data("luquillo_tree5_random") %>%
  subset(sp %in% c("PREMON", "CASARB"))

autoplot(sp(census))

# Customize
autoplot(sp(census), point_size = 1)

# Elevation -------------------------------------------------------------

elevation <- fgeo.x::elevation
autoplot(elev(elevation))
## Not run: 
# Same: Works both with the elevation list and dataframe
autoplot(elev(elevation$col))

# Customize
autoplot(elev(elevation), contour_size = 1)

## End(Not run)

# Species and elevation -------------------------------------------------

autoplot(sp_elev(census, elevation), facet = FALSE, point_size = 1)
\end{ExampleCode}
\end{Examples}
\inputencoding{utf8}
\HeaderA{autoplot\_by\_species.sp\_elev}{List plots of species distribution and topography (good for pdf output).}{autoplot.Rul.by.Rul.species.sp.Rul.elev}
\aliasA{autoplot\_by\_species.sp}{autoplot\_by\_species.sp\_elev}{autoplot.Rul.by.Rul.species.sp}
%
\begin{Description}\relax
These functions extend \code{\LinkA{autoplot.sp()}{autoplot.sp}} and \code{\LinkA{autoplot.elev()}{autoplot.elev}} and return not a
single plot but a list of plots. They are particularly useful if you want to
print a \emph{.pdf} file with one plot per page. They automatically plot the
variables \code{sp} and \code{elev} of a ForestGEO-like dataset of class 'sp' or
'sp\_elev'.
\begin{itemize}

\item Create a 'sp' \code{object} with:

\end{itemize}


\begin{alltt}    object <- sp(DATA-WITH-VARIABLE-sp)
\end{alltt}

\begin{itemize}

\item Create a 'sp\_elev' \code{object} with:

\end{itemize}


\begin{alltt}    object <- sp_elev(DATA-WITH-VARIABLE-sp, DATA-WITH-VARIABLE-elev)
\end{alltt}


See sections \strong{Usage} and \strong{Examples}.
\end{Description}
%
\begin{Usage}
\begin{verbatim}
## S3 method for class 'sp_elev'
autoplot_by_species(object, species = "all",
  fill = "black", shape = 21, point_size = 3, contour_size = 0.5,
  low = "blue", high = "red", hide_color_legend = FALSE,
  bins = NULL, add_elevation_labels = TRUE, label_size = 3,
  label_color = "grey", xyjust = 1, fontface = "italic",
  xlim = NULL, ylim = NULL, custom_theme = NULL, ...)

## S3 method for class 'sp'
autoplot_by_species(object, species = "all",
  fill = "black", shape = 21, point_size = 3,
  hide_color_legend = FALSE, xlim = NULL, ylim = NULL,
  custom_theme = NULL, ...)
\end{verbatim}
\end{Usage}
%
\begin{Arguments}
\begin{ldescription}
\item[\code{object}] An object created with \code{\LinkA{sp()}{sp}} or \code{\LinkA{sp\_elev()}{sp.Rul.elev}}.

\item[\code{species}] A character vector giving values in the column \code{sp}. The
output will be a list with as many plots as elements in this vector.
The string "all" (default) plots all unique values of \code{sp}.

\item[\code{fill}] Character; either a color or "sp", which maps each species to a
different color.

\item[\code{shape}] A number giving point shape (as in \code{\LinkA{graphics::points()}{graphics::points()}}). Passed
to \code{\LinkA{ggplot2::geom\_point()}{ggplot2::geom.Rul.point()}}.

\item[\code{point\_size}] A number giving point size. Passed to
\code{\LinkA{ggplot2::geom\_point()}{ggplot2::geom.Rul.point()}}.

\item[\code{contour\_size}] A number giving the size of the contour of elevation
lines. Passed to \code{ggplot2::stat\_contour()} (see \code{\LinkA{ggplot2::geom\_contour()}{ggplot2::geom.Rul.contour()}}).

\item[\code{low, high}] A string giving a color of the elevation lines representing
low and high elevation.

\item[\code{hide\_color\_legend}] Logical; \code{TRUE} hides the color legend.

\item[\code{bins}] A number giving the number of elevation lines to plot.

\item[\code{add\_elevation\_labels}] Logical; \code{FALSE} hides elevation labels.

\item[\code{label\_size, label\_color, fontface}] A number (\code{label\_size}) or character
string (\code{label\_color} and \code{fontface}) giving the size, colour and fontface
of the text labels for the elevation lines.

\item[\code{xyjust}] A number to adjust the position of the text labels of the
elevation lines.

\item[\code{xlim, ylim}] A vector of length 2, for example \code{c(0, 500)}, giving the
minimum and maximum limits of the vertical and horizontal coordinates.

\item[\code{custom\_theme}] A valid \code{\LinkA{ggplot2::theme()}{ggplot2::theme()}}. \code{NULL} uses the default
theme \code{\LinkA{theme\_default()}{theme.Rul.default}}.

\item[\code{...}] Not used (included for compatibility across methods).
\end{ldescription}
\end{Arguments}
%
\begin{Details}\relax
\code{autoplot\_by\_species(sp\_elev(DATA-WITH-VARIABLE-sp)} (without elevation data)
is equivalent to \code{autoplot\_by\_species(sp(DATA-WITH-VARIABLE-sp))}.

\strong{fgeo.plot} wraps some functions from the \strong{ggplot2} package.
For more control you can use \strong{ggplot2} directly.
\end{Details}
%
\begin{Value}
A list of objects of class "ggplot".
\end{Value}
%
\begin{SeeAlso}\relax
\code{\LinkA{autoplot()}{autoplot}}, \code{\LinkA{sp()}{sp}}, \code{\LinkA{sp\_elev()}{sp.Rul.elev}}.

Other plot functions: \code{\LinkA{autoplot.fgeo\_habitat}{autoplot.fgeo.Rul.habitat}},
\code{\LinkA{autoplot.sp\_elev}{autoplot.sp.Rul.elev}}, \code{\LinkA{elev}{elev}},
\code{\LinkA{plot\_dbh\_bubbles\_by\_quadrat}{plot.Rul.dbh.Rul.bubbles.Rul.by.Rul.quadrat}},
\code{\LinkA{plot\_tag\_status\_by\_subquadrat}{plot.Rul.tag.Rul.status.Rul.by.Rul.subquadrat}},
\code{\LinkA{sp\_elev}{sp.Rul.elev}}, \code{\LinkA{sp}{sp}}

Other functions to plot elevation: \code{\LinkA{autoplot.sp\_elev}{autoplot.sp.Rul.elev}},
\code{\LinkA{elev}{elev}}, \code{\LinkA{sp\_elev}{sp.Rul.elev}}

Other functions to plot species: \code{\LinkA{autoplot.sp\_elev}{autoplot.sp.Rul.elev}},
\code{\LinkA{sp\_elev}{sp.Rul.elev}}, \code{\LinkA{sp}{sp}}
\end{SeeAlso}
%
\begin{Examples}
\begin{ExampleCode}
assert_is_installed("fgeo.x")

# Species ---------------------------------------------------------------
# Small dataset with a few species for quick examples
census <- fgeo.x::tree6_3species

# Showing only two species for speed
autoplot_by_species(sp(census))[1:2]
## Not run: 
# Print all plots in the list
pdf("map.pdf", paper = "letter", height = 10.5, width = 8)
autoplot_by_species(sp(census))
dev.off()

## End(Not run)

# Species and elevation (optional) ---------------------------------------
# Species only: Same as autoplot_by_species(sp(census))
autoplot_by_species(sp_elev(census))

# Species and elevation
elevation <- fgeo.x::elevation
autoplot_by_species(sp_elev(census, elevation))
\end{ExampleCode}
\end{Examples}
\inputencoding{utf8}
\HeaderA{elev}{Allow autoplotting the column \code{elev}.}{elev}
%
\begin{Description}\relax
Allow autoplotting the column \code{elev}.
\end{Description}
%
\begin{Usage}
\begin{verbatim}
elev(elev)
\end{verbatim}
\end{Usage}
%
\begin{Arguments}
\begin{ldescription}
\item[\code{elev}] A ForestGEO-like elevation list or its \code{col} dataframe (with
the column \code{elev}).
\end{ldescription}
\end{Arguments}
%
\begin{Value}
An S3 object of class 'elev'.
\end{Value}
%
\begin{SeeAlso}\relax
\code{\LinkA{autoplot.elev()}{autoplot.elev}}.

Other plot functions: \code{\LinkA{autoplot.fgeo\_habitat}{autoplot.fgeo.Rul.habitat}},
\code{\LinkA{autoplot.sp\_elev}{autoplot.sp.Rul.elev}},
\code{\LinkA{autoplot\_by\_species.sp\_elev}{autoplot.Rul.by.Rul.species.sp.Rul.elev}},
\code{\LinkA{plot\_dbh\_bubbles\_by\_quadrat}{plot.Rul.dbh.Rul.bubbles.Rul.by.Rul.quadrat}},
\code{\LinkA{plot\_tag\_status\_by\_subquadrat}{plot.Rul.tag.Rul.status.Rul.by.Rul.subquadrat}},
\code{\LinkA{sp\_elev}{sp.Rul.elev}}, \code{\LinkA{sp}{sp}}

Other autoplots: \code{\LinkA{autoplot.fgeo\_habitat}{autoplot.fgeo.Rul.habitat}},
\code{\LinkA{autoplot.sp\_elev}{autoplot.sp.Rul.elev}}, \code{\LinkA{sp\_elev}{sp.Rul.elev}},
\code{\LinkA{sp}{sp}}

Other functions to construct fgeo classes: \code{\LinkA{sp\_elev}{sp.Rul.elev}},
\code{\LinkA{sp}{sp}}

Other functions to plot elevation: \code{\LinkA{autoplot.sp\_elev}{autoplot.sp.Rul.elev}},
\code{\LinkA{autoplot\_by\_species.sp\_elev}{autoplot.Rul.by.Rul.species.sp.Rul.elev}},
\code{\LinkA{sp\_elev}{sp.Rul.elev}}
\end{SeeAlso}
%
\begin{Examples}
\begin{ExampleCode}
assert_is_installed("fgeo.x")

inherits(elev(fgeo.x::elevation), "elev")
inherits(elev(fgeo.x::elevation$col), "elev")
\end{ExampleCode}
\end{Examples}
\inputencoding{utf8}
\HeaderA{plot\_dbh\_bubbles\_by\_quadrat}{List dbh bubble-plots by quadrat (good for .pdf output).}{plot.Rul.dbh.Rul.bubbles.Rul.by.Rul.quadrat}
%
\begin{Description}\relax
List dbh bubble-plots by quadrat (good for .pdf output).
\end{Description}
%
\begin{Usage}
\begin{verbatim}
plot_dbh_bubbles_by_quadrat(vft,
  title_quad = "Site Name, YYYY, Quadrat:",
  header = header_dbh_bubbles(), theme = theme_dbh_bubbles(),
  lim_min = 0, lim_max = 20, subquadrat_side = 5, tag_size = 2,
  move_edge = 0, status_d = "dead")
\end{verbatim}
\end{Usage}
%
\begin{Arguments}
\begin{ldescription}
\item[\code{vft}] A ForestGEO ViewFullTable (dataframe).

\item[\code{title\_quad}] A string to use as a title.

\item[\code{header}] A string to use as a header (see \LinkA{headers}{headers}).

\item[\code{theme}] An object of class "theme".

\item[\code{lim\_min, lim\_max}] Minimum and maximum limits of the plot area.

\item[\code{subquadrat\_side}] Length in meters of the side of a subquadrat.

\item[\code{tag\_size}] A number giving tag size. Passed to
\LinkA{ggrepel::geom\_text\_repel}{ggrepel::geom.Rul.text.Rul.repel}.

\item[\code{move\_edge}] A number to adjust the extension of the grid lines beyond
the plot limits.

\item[\code{status\_d}] A character string indicating the value of the variable
status that corresponds to dead stems.
\end{ldescription}
\end{Arguments}
%
\begin{Value}
A list which each element is a plot of class ggplot.
\end{Value}
%
\begin{SeeAlso}\relax
Other plot functions: \code{\LinkA{autoplot.fgeo\_habitat}{autoplot.fgeo.Rul.habitat}},
\code{\LinkA{autoplot.sp\_elev}{autoplot.sp.Rul.elev}},
\code{\LinkA{autoplot\_by\_species.sp\_elev}{autoplot.Rul.by.Rul.species.sp.Rul.elev}},
\code{\LinkA{elev}{elev}},
\code{\LinkA{plot\_tag\_status\_by\_subquadrat}{plot.Rul.tag.Rul.status.Rul.by.Rul.subquadrat}},
\code{\LinkA{sp\_elev}{sp.Rul.elev}}, \code{\LinkA{sp}{sp}}

Other functions to list plots from ForestGEO ViewFullTable: \code{\LinkA{plot\_tag\_status\_by\_subquadrat}{plot.Rul.tag.Rul.status.Rul.by.Rul.subquadrat}}

Other functions to plot dbh bubbles: \code{\LinkA{header\_dbh\_bubbles}{header.Rul.dbh.Rul.bubbles}},
\code{\LinkA{theme\_dbh\_bubbles}{theme.Rul.dbh.Rul.bubbles}}
\end{SeeAlso}
%
\begin{Examples}
\begin{ExampleCode}
assert_is_installed("fgeo.x")

# Create a small VieFullTable
first_n <- function(x, n) x %in% sort(unique(x))[1:n]
small_vft <- fgeo.x::vft_4quad %>%
  dplyr::filter(first_n(CensusID, 1) & first_n(QuadratID, 2)) %>%
  dplyr::sample_n(50)

plot_dbh_bubbles_by_quadrat(small_vft)

# Printing all plots to .pdf, with parameters optimized for size letter
## Not run: 
pdf("map.pdf", paper = "letter", height = 10.5, width = 8)
plot_dbh_bubbles_by_quadrat(small_vft)
dev.off()

## End(Not run)

# Be careful if subsetting by DBH: You may unintentionally remove dead trees.
# You should explicietly inlcude missing `DBH` values with `is.na(DBH)`
include_missing_dbh <- subset(small_vft, DBH > 20 | is.na(DBH))
plot_dbh_bubbles_by_quadrat(include_missing_dbh)

# Customizing the maps ----------------------------------------------------
# A custom title and header
myheader <- paste(
  " ",
  "Head column 1                     Head column 2                          ",
  " ",
  " ........................................................................",
  " ........................................................................",
  sep = "\n"
)

plot_dbh_bubbles_by_quadrat(
  small_vft,
  title_quad = "My Site, 2018. Quad:",
  header = myheader
)

# Tweak the theme with ggplot
library(ggplot2)

plot_dbh_bubbles_by_quadrat(
  small_vft,
  title_quad = "My Site, 2018. Quad:",
  header = header_dbh_bubbles("spanish"),
  tag_size = 3,
  theme = theme_dbh_bubbles(
    axis.text = NULL, # NULL shows axis.text; element_blank() doesn't.
    plot.title = element_text(size = 15),
    plot.subtitle = element_text(size = 5),
    panel.background = element_rect(fill = "grey")
  )
)
\end{ExampleCode}
\end{Examples}
\inputencoding{utf8}
\HeaderA{plot\_tag\_status\_by\_subquadrat}{List plots of tree-tag status by subquadrat (good for .pdf output).}{plot.Rul.tag.Rul.status.Rul.by.Rul.subquadrat}
%
\begin{Description}\relax
This function plots tree tags by status and outputs a list of plots that can
be printed on a .pdf file. Each plot shows four subquadrats within a quadrat.
The symbols on the plot represent the status of each tree -- not the status of
each stem. Although you should likely provide data of only one or two
censuses, \code{plot\_tag\_status\_by\_subquadrat()} will summarize the data to reduce
overplotting. The data on the plot summarizes the history of each stem across
all censuses provided. Each tag will appear in the plot only once or twice:
\begin{itemize}

\item A tag will appear once if it belongs to a tree which status was unique
across all censuses provided -- either "alive" or "dead".
\item A tag will appear twice if it belongs to a tree which status was "alive" in
at least one census, and also "dead" in at least one other census. This
feature avoids unintentional overplotting and makes interpreting the plot
easier.

\end{itemize}

\end{Description}
%
\begin{Usage}
\begin{verbatim}
plot_tag_status_by_subquadrat(vft, x_q = 20, x_sq = 5, y_q = 20,
  y_sq = 5, subquad_offset = NULL, bl = 1, br = 2, tr = 3,
  tl = 4, title_quad = "Site Name, YYYY. Quadrat:", show_page = TRUE,
  show_subquad = TRUE, point_shape = c(19, 4), point_size = 1.5,
  tag_size = 3, header = header_tag_status(),
  theme = theme_tag_status(), move_edge = 0)
\end{verbatim}
\end{Usage}
%
\begin{Arguments}
\begin{ldescription}
\item[\code{vft}] A ForestGEO ViewFullTable (dataframe).

\item[\code{x\_q, y\_q}] Size in meters of a quadrat's side. For ForestGEO sites, a
common value is 20.

\item[\code{x\_sq, y\_sq}] Size in meters of a subquadrat's side. For ForestGEO-CTFS
sites, a common value is 5.

\item[\code{subquad\_offset}] \code{NULL} or \code{-1}. \code{NULL} defines the first column of
subquadrats as 1.  \code{-1} defines the first column of subquadrats as 0.\begin{alltt}subquad_offset = NULL    subquad_offset = -1
---------------------    -------------------
     14 24 34 44             04 14 24 34
     13 23 33 43             03 13 23 33
     12 22 32 42             02 12 22 32
     11 21 31 41             01 11 21 31
\end{alltt}


\item[\code{bl, br, tr, tl}] Number or character giving the label of the four
subquadrats on each or the four divisions of a quadrat: bottom left (bl),
bottom right (br), top right (tr), and top left (tl).

\item[\code{title\_quad}] A string to use as a title.

\item[\code{show\_page}] Logical; \code{FALSE} removes the page label from the plot title.

\item[\code{show\_subquad}] Logical; \code{FALSE} removes subquadrat labels on each plot.

\item[\code{point\_shape}] A vector of two numbers giving the shape of the points to
plot (see possible shapes in the documentation of ?\code{\LinkA{graphics::points()}{graphics::points()}},
under the section entitled \emph{'pch' values}).

\item[\code{point\_size}] A number giving points size. Passed to
\code{\LinkA{ggplot2::geom\_point()}{ggplot2::geom.Rul.point()}}.

\item[\code{tag\_size}] A number giving tag size. Passed to
\LinkA{ggrepel::geom\_text\_repel}{ggrepel::geom.Rul.text.Rul.repel}.

\item[\code{header}] A string to use as a header (see \LinkA{headers}{headers}).

\item[\code{theme}] An object of class "theme".

\item[\code{move\_edge}] A number to adjust the extension of the grid lines beyond
the plot limits.
\end{ldescription}
\end{Arguments}
%
\begin{Value}
A list of objects of class "ggplot".
\end{Value}
%
\begin{Section}{Acknowledgment}

Useful ideas and guidance came from Suzanne Lao, Stuart Davis, Shameema
Jafferjee Esufali, David Kenfack and Anudeep Singh. Anudeep Sinh also wrote
the algorithm to calculate subquadrats.
\end{Section}
%
\begin{SeeAlso}\relax
\code{\LinkA{graphics::points()}{graphics::points()}}, \code{\LinkA{ggplot2::geom\_point()}{ggplot2::geom.Rul.point()}}, \code{\LinkA{ggplot2::theme()}{ggplot2::theme()}}
\code{\LinkA{header\_tag\_status()}{header.Rul.tag.Rul.status}}, \code{\LinkA{theme\_tag\_status()}{theme.Rul.tag.Rul.status}}, \code{\LinkA{fgeo.tool::add\_subquad()}{fgeo.tool::add.Rul.subquad()}},
\LinkA{ggrepel::geom\_text\_repel}{ggrepel::geom.Rul.text.Rul.repel}.

Other plot functions: \code{\LinkA{autoplot.fgeo\_habitat}{autoplot.fgeo.Rul.habitat}},
\code{\LinkA{autoplot.sp\_elev}{autoplot.sp.Rul.elev}},
\code{\LinkA{autoplot\_by\_species.sp\_elev}{autoplot.Rul.by.Rul.species.sp.Rul.elev}},
\code{\LinkA{elev}{elev}},
\code{\LinkA{plot\_dbh\_bubbles\_by\_quadrat}{plot.Rul.dbh.Rul.bubbles.Rul.by.Rul.quadrat}},
\code{\LinkA{sp\_elev}{sp.Rul.elev}}, \code{\LinkA{sp}{sp}}

Other functions to list plots from ForestGEO ViewFullTable: \code{\LinkA{plot\_dbh\_bubbles\_by\_quadrat}{plot.Rul.dbh.Rul.bubbles.Rul.by.Rul.quadrat}}

Other functions to plot tag status: \code{\LinkA{header\_tag\_status}{header.Rul.tag.Rul.status}},
\code{\LinkA{theme\_tag\_status}{theme.Rul.tag.Rul.status}}
\end{SeeAlso}
%
\begin{Examples}
\begin{ExampleCode}
assert_is_installed("fgeo.x")

# Create a small VieFullTable
first <- function(x) x %in% sort(unique(x))[1]
small_vft <- subset(fgeo.x::vft_4quad, first(CensusID) & first(QuadratID))

p <- plot_tag_status_by_subquadrat(small_vft)
# Showing only two sub-quadtrats
p[1:2]

# Print all plots to .pdf, with parameters optimized for size letter
## Not run: 
pdf("map.pdf", paper = "letter", height = 10.5, width = 8)
plot_tag_status_by_subquadrat(small_vft)
dev.off()

## End(Not run)

# Be careful if filtering by DBH: You may unintentionally remove dead trees.
# * If you filter by `DBH`, you loose the dead trees becaue their `DBH = NA`
# * You should explicietly inlcude missing DBH values with `is.na(DBH)`
include_missing_dbh <- subset(small_vft, DBH > 20 | is.na(DBH))
p <- plot_tag_status_by_subquadrat(include_missing_dbh)
# Showing only the first plot to keep the output short
p[[1]]

# Customizing the maps ----------------------------------------------------
# Common tweaks
p <- plot_tag_status_by_subquadrat(
  small_vft,
  show_page = FALSE,
  show_subquad = FALSE
)
p[[1]]

p <- plot_tag_status_by_subquadrat(
  small_vft,
  title_quad = "BCI 2012. Quadrat: ",
  bl = "bottom-left", br = "bottom-right", tr = "top-right", tl = "top-left",
  header = "Line 1: _________\nLine 2:\nLine 3:.....................",
  subquad_offset = -1,
  point_size = 3, point_shape = c(17, 6),
  tag_size = 2,
  move_edge = 0.5
)
p[[1]]

# Themes
library(ggplot2)

p <- plot_tag_status_by_subquadrat(small_vft, theme = theme_gray())
p[[1]]

# Tweaking the default theme of plot_tag_status_by_subquadrat()
# For many more options see ?ggplot2::theme
small_tweak <- theme_tag_status(legend.position = "bottom")
p <- plot_tag_status_by_subquadrat(small_vft, theme = small_tweak)
p[[1]]
\end{ExampleCode}
\end{Examples}
\inputencoding{utf8}
\HeaderA{sp}{Allow autoplotting the column \code{sp}.}{sp}
%
\begin{Description}\relax
Allow autoplotting the column \code{sp}.
\end{Description}
%
\begin{Usage}
\begin{verbatim}
sp(sp)
\end{verbatim}
\end{Usage}
%
\begin{Arguments}
\begin{ldescription}
\item[\code{sp}] A ForestGEO-like dataframe with the column \code{sp}.
\end{ldescription}
\end{Arguments}
%
\begin{Value}
An S3 object of class 'sp'.
\end{Value}
%
\begin{SeeAlso}\relax
\code{\LinkA{autoplot.sp()}{autoplot.sp}}.

Other plot functions: \code{\LinkA{autoplot.fgeo\_habitat}{autoplot.fgeo.Rul.habitat}},
\code{\LinkA{autoplot.sp\_elev}{autoplot.sp.Rul.elev}},
\code{\LinkA{autoplot\_by\_species.sp\_elev}{autoplot.Rul.by.Rul.species.sp.Rul.elev}},
\code{\LinkA{elev}{elev}},
\code{\LinkA{plot\_dbh\_bubbles\_by\_quadrat}{plot.Rul.dbh.Rul.bubbles.Rul.by.Rul.quadrat}},
\code{\LinkA{plot\_tag\_status\_by\_subquadrat}{plot.Rul.tag.Rul.status.Rul.by.Rul.subquadrat}},
\code{\LinkA{sp\_elev}{sp.Rul.elev}}

Other autoplots: \code{\LinkA{autoplot.fgeo\_habitat}{autoplot.fgeo.Rul.habitat}},
\code{\LinkA{autoplot.sp\_elev}{autoplot.sp.Rul.elev}}, \code{\LinkA{elev}{elev}},
\code{\LinkA{sp\_elev}{sp.Rul.elev}}

Other functions to construct fgeo classes: \code{\LinkA{elev}{elev}},
\code{\LinkA{sp\_elev}{sp.Rul.elev}}

Other functions to plot species: \code{\LinkA{autoplot.sp\_elev}{autoplot.sp.Rul.elev}},
\code{\LinkA{autoplot\_by\_species.sp\_elev}{autoplot.Rul.by.Rul.species.sp.Rul.elev}},
\code{\LinkA{sp\_elev}{sp.Rul.elev}}
\end{SeeAlso}
%
\begin{Examples}
\begin{ExampleCode}
assert_is_installed("fgeo.x")

inherits(sp(fgeo.x::stem5), "sp")
\end{ExampleCode}
\end{Examples}
\inputencoding{utf8}
\HeaderA{sp\_elev}{Allow autoplotting the columns \code{sp} and \code{elev}.}{sp.Rul.elev}
%
\begin{Description}\relax
Allow autoplotting the columns \code{sp} and \code{elev}.
\end{Description}
%
\begin{Usage}
\begin{verbatim}
sp_elev(sp, elev = NULL)
\end{verbatim}
\end{Usage}
%
\begin{Arguments}
\begin{ldescription}
\item[\code{sp}] A ForestGEO-like dataframe with the column \code{sp}.

\item[\code{elev}] A ForestGEO-like elevation list or its \code{col} dataframe -- with
the column \code{elev}.
\end{ldescription}
\end{Arguments}
%
\begin{Value}
An S3 object of class 'sp\_elev'.
\end{Value}
%
\begin{SeeAlso}\relax
\code{\LinkA{autoplot.sp\_elev()}{autoplot.sp.Rul.elev}}.

Other plot functions: \code{\LinkA{autoplot.fgeo\_habitat}{autoplot.fgeo.Rul.habitat}},
\code{\LinkA{autoplot.sp\_elev}{autoplot.sp.Rul.elev}},
\code{\LinkA{autoplot\_by\_species.sp\_elev}{autoplot.Rul.by.Rul.species.sp.Rul.elev}},
\code{\LinkA{elev}{elev}},
\code{\LinkA{plot\_dbh\_bubbles\_by\_quadrat}{plot.Rul.dbh.Rul.bubbles.Rul.by.Rul.quadrat}},
\code{\LinkA{plot\_tag\_status\_by\_subquadrat}{plot.Rul.tag.Rul.status.Rul.by.Rul.subquadrat}},
\code{\LinkA{sp}{sp}}

Other autoplots: \code{\LinkA{autoplot.fgeo\_habitat}{autoplot.fgeo.Rul.habitat}},
\code{\LinkA{autoplot.sp\_elev}{autoplot.sp.Rul.elev}}, \code{\LinkA{elev}{elev}},
\code{\LinkA{sp}{sp}}

Other functions to construct fgeo classes: \code{\LinkA{elev}{elev}},
\code{\LinkA{sp}{sp}}

Other functions to plot elevation: \code{\LinkA{autoplot.sp\_elev}{autoplot.sp.Rul.elev}},
\code{\LinkA{autoplot\_by\_species.sp\_elev}{autoplot.Rul.by.Rul.species.sp.Rul.elev}},
\code{\LinkA{elev}{elev}}

Other functions to plot species: \code{\LinkA{autoplot.sp\_elev}{autoplot.sp.Rul.elev}},
\code{\LinkA{autoplot\_by\_species.sp\_elev}{autoplot.Rul.by.Rul.species.sp.Rul.elev}},
\code{\LinkA{sp}{sp}}
\end{SeeAlso}
%
\begin{Examples}
\begin{ExampleCode}
assert_is_installed("fgeo.x")

species_elevation <- sp_elev(fgeo.x::stem5, fgeo.x::elevation)
inherits(species_elevation, "sp_elev")
\end{ExampleCode}
\end{Examples}
\printindex{}
\end{document}
