\nonstopmode{}
\documentclass[a4paper]{book}
\usepackage[times,inconsolata,hyper]{Rd}
\usepackage{makeidx}
\usepackage[utf8]{inputenc} % @SET ENCODING@
% \usepackage{graphicx} % @USE GRAPHICX@
\makeindex{}
\begin{document}
\chapter*{}
\begin{center}
{\textbf{\huge Package `fgeo.tool'}}
\par\bigskip{\large \today}
\end{center}
\begin{description}
\raggedright{}
\inputencoding{utf8}
\item[Title]\AsIs{Import and Manipulate ForestGEO Data}
\item[Version]\AsIs{1.2.1}
\item[Description]\AsIs{Import and manipulate ForestGEO data
(<http://www.forestgeo.si.edu/>).}
\item[License]\AsIs{GPL-3}
\item[URL]\AsIs{}\url{https://github.com/forestgeo/fgeo.tool}\AsIs{}
\item[BugReports]\AsIs{}\url{https://github.com/forestgeo/fgeo.tool/issues}\AsIs{}
\item[Depends]\AsIs{R (>= 3.3)}
\item[Imports]\AsIs{dplyr (>= 0.7.8), glue (>= 1.3.0), magrittr (>= 1.5), purrr
(>= 0.3.0), readr (>= 1.3.1), rlang (>= 0.3.1), tibble (>=
2.0.1), tidyselect (>= 0.2.5)}
\item[Suggests]\AsIs{covr (>= 3.2.1), fgeo.x (>= 1.1.0), knitr (>= 1.21), pingr
(>= 1.1.2), roxygen2 (>= 6.1.1), spelling (>= 2.0), stringr (>=
1.3.1), testthat (>= 2.0.1), tidyr (>= 0.8.2)}
\item[Additional\_repositories]\AsIs{https://forestgeo.github.io/drat/}
\item[Encoding]\AsIs{UTF-8}
\item[Language]\AsIs{en-US}
\item[LazyData]\AsIs{true}
\item[Roxygen]\AsIs{list(markdown = TRUE)}
\item[RoxygenNote]\AsIs{6.1.1}
\item[NeedsCompilation]\AsIs{no}
\item[Author]\AsIs{Mauro Lepore [aut, ctr, cre],
Richard Condit [aut],
Suzanne Lao [aut],
CTFS-ForestGEO [cph, fnd]}
\item[Maintainer]\AsIs{Mauro Lepore }\email{leporem@si.edu}\AsIs{}
\end{description}
\Rdcontents{\R{} topics documented:}
\inputencoding{utf8}
\HeaderA{add\_status\_tree}{Add column \code{status\_tree} based on the status of all stems of each tree.}{add.Rul.status.Rul.tree}
%
\begin{Description}\relax
Add column \code{status\_tree} based on the status of all stems of each tree.
\end{Description}
%
\begin{Usage}
\begin{verbatim}
add_status_tree(data, status_a = "A", status_d = "D")
\end{verbatim}
\end{Usage}
%
\begin{Arguments}
\begin{ldescription}
\item[\code{data}] A ForestGEO-like dataframe: A \emph{ViewFullTable}, \emph{tree} or \emph{stem}
table.

\item[\code{status\_a, status\_d}] Sting to match alive and dead stems; it corresponds
to the values of the variable \code{status} (in census tables) or \code{Status} (with
capital "S" in ViewFull tables).
\end{ldescription}
\end{Arguments}
%
\begin{Value}
The input data set with the additional variable \code{status\_tree}.
\end{Value}
%
\begin{SeeAlso}\relax
Other functions to add columns to dataframes: \code{\LinkA{add\_subquad}{add.Rul.subquad}},
\code{\LinkA{add\_var}{add.Rul.var}}

Other functions for ForestGEO data: \code{\LinkA{add\_subquad}{add.Rul.subquad}},
\code{\LinkA{add\_var}{add.Rul.var}}

Other functions for fgeo census: \code{\LinkA{add\_var}{add.Rul.var}},
\code{\LinkA{guess\_plotdim}{guess.Rul.plotdim}}, \code{\LinkA{pick\_drop}{pick.Rul.drop}}

Other functions for fgeo vft: \code{\LinkA{add\_subquad}{add.Rul.subquad}},
\code{\LinkA{add\_var}{add.Rul.var}}, \code{\LinkA{guess\_plotdim}{guess.Rul.plotdim}},
\code{\LinkA{pick\_drop}{pick.Rul.drop}}
\end{SeeAlso}
%
\begin{Examples}
\begin{ExampleCode}
stem <- tribble(
  ~CensusID, ~treeID, ~stemID, ~status,
          1,       1,       1,     "A",
          1,       1,       2,     "D",
          
          1,       2,       3,     "D",
          1,       2,       4,     "D",
          
          
          
          2,       1,       1,     "A",
          2,       1,       2,     "G",
          
          2,       2,       3,     "D",
          2,       2,       4,     "G"
)

add_status_tree(stem)

\end{ExampleCode}
\end{Examples}
\inputencoding{utf8}
\HeaderA{add\_subquad}{Add column \code{subquadrat} based on \code{QX} and \code{QY} coordinates.}{add.Rul.subquad}
%
\begin{Description}\relax
Add column \code{subquadrat} based on \code{QX} and \code{QY} coordinates.
\end{Description}
%
\begin{Usage}
\begin{verbatim}
add_subquad(data, x_q, y_q = x_q, x_sq, y_sq = x_sq,
  subquad_offset = NULL)
\end{verbatim}
\end{Usage}
%
\begin{Arguments}
\begin{ldescription}
\item[\code{data}] A dataframe with quadrat coordinates \code{QX} and \code{QY} (e.g. a
\emph{ViewFullTable}).

\item[\code{x\_q, y\_q}] Size in meters of a quadrat's side. For ForestGEO sites, a
common value is 20.

\item[\code{x\_sq, y\_sq}] Size in meters of a subquadrat's side. For ForestGEO sites,
a common value is 5.

\item[\code{subquad\_offset}] Either \code{-1} or \code{1}, to rest or add one unit to the
digit of each subquadrat that represents the column number.\begin{alltt}First column is 0    First column is 1
-----------------    -----------------
   04 14 24 34          14 24 34 44
   03 13 23 33          13 23 33 43
   02 12 22 32          12 22 32 42
   01 11 21 31          11 21 31 41
\end{alltt}

\end{ldescription}
\end{Arguments}
%
\begin{Value}
Returns \code{data} with the additional variable \code{subquadrat}.
\end{Value}
%
\begin{Author}\relax
Anudeep Singh and Mauro Lepore.
\end{Author}
%
\begin{SeeAlso}\relax
Other functions to add columns to dataframes: \code{\LinkA{add\_status\_tree}{add.Rul.status.Rul.tree}},
\code{\LinkA{add\_var}{add.Rul.var}}

Other functions for ForestGEO data: \code{\LinkA{add\_status\_tree}{add.Rul.status.Rul.tree}},
\code{\LinkA{add\_var}{add.Rul.var}}

Other functions for fgeo vft: \code{\LinkA{add\_status\_tree}{add.Rul.status.Rul.tree}},
\code{\LinkA{add\_var}{add.Rul.var}}, \code{\LinkA{guess\_plotdim}{guess.Rul.plotdim}},
\code{\LinkA{pick\_drop}{pick.Rul.drop}}
\end{SeeAlso}
%
\begin{Examples}
\begin{ExampleCode}
vft <- tribble(
   ~QX,  ~QY,
  17.9,    0,
   4.1,   15,
   6.1, 17.3,
   3.8,  5.9,
   4.5, 12.4,
   4.9,  9.3,
   9.8,  3.2,
  18.6,  1.1,
  17.3,  4.1,
   1.5, 16.3
)

add_subquad(vft, 20, 20, 5, 5)

add_subquad(vft, 20, 20, 5, 5, subquad_offset = -1)

\end{ExampleCode}
\end{Examples}
\inputencoding{utf8}
\HeaderA{add\_var}{Add columns \code{lx/ly}, \code{QX/QY}, \code{index}, \code{col/row}, \code{hectindex}, \code{quad}, \code{gx/gy}.}{add.Rul.var}
\aliasA{add\_col\_row}{add\_var}{add.Rul.col.Rul.row}
\aliasA{add\_gxgy}{add\_var}{add.Rul.gxgy}
\aliasA{add\_hectindex}{add\_var}{add.Rul.hectindex}
\aliasA{add\_index}{add\_var}{add.Rul.index}
\aliasA{add\_lxly}{add\_var}{add.Rul.lxly}
\aliasA{add\_quad}{add\_var}{add.Rul.quad}
\aliasA{add\_qxqy}{add\_var}{add.Rul.qxqy}
%
\begin{Description}\relax
These functions add columns to position trees in a forest plot. They work
with \emph{ViewFullTable}, \emph{tree} and \emph{stem} tables. From the input table, most
functions use only the \code{gx} and \code{gy} columns (or equivalent columns). The
exception is the function \code{add\_gxgy()} which inputs quadrat information. If
your data lacks some important column, an error message will inform you which
column is missing.
\end{Description}
%
\begin{Usage}
\begin{verbatim}
add_lxly(data, gridsize = 20, plotdim = NULL)

add_qxqy(data, gridsize = 20, plotdim = NULL)

add_index(data, gridsize = 20, plotdim = NULL)

add_col_row(data, gridsize = 20, plotdim = NULL)

add_hectindex(data, gridsize = 20, plotdim = NULL)

add_quad(data, gridsize = 20, plotdim = NULL, start = NULL,
  width = 2)

add_gxgy(data, gridsize = 20, start = 0)
\end{verbatim}
\end{Usage}
%
\begin{Arguments}
\begin{ldescription}
\item[\code{data}] A ForestGEO-like dataframe: A \emph{ViewFullTable}, \emph{tree} or \emph{stem}
table.

\item[\code{gridsize}] The gridsize of the census plot (commonly 20 m).

\item[\code{plotdim}] The global dimensions of the census plot (i.e. the
maximum possible values of \code{gx} and \code{gy}).

\item[\code{start}] Defaults to label the first quadrat as "0101". Use \code{0} to
label it as "0000" instead.

\item[\code{width}] Number; width to pad the labels of plot-columns and -rows.
\end{ldescription}
\end{Arguments}
%
\begin{Details}\relax
These functions are adapted from the \Rhref{http://ctfs.si.edu/Public/CTFSRPackage/}{CTFS R Package}.
\end{Details}
%
\begin{Value}
For any given \code{var}, a function \code{add\_var()} returns a modified
version of the input dataframe, with the additional variable(s) \code{var}.
\end{Value}
%
\begin{SeeAlso}\relax
Other functions to add columns to dataframes: \code{\LinkA{add\_status\_tree}{add.Rul.status.Rul.tree}},
\code{\LinkA{add\_subquad}{add.Rul.subquad}}

Other functions for ForestGEO data: \code{\LinkA{add\_status\_tree}{add.Rul.status.Rul.tree}},
\code{\LinkA{add\_subquad}{add.Rul.subquad}}

Other functions for fgeo census: \code{\LinkA{add\_status\_tree}{add.Rul.status.Rul.tree}},
\code{\LinkA{guess\_plotdim}{guess.Rul.plotdim}}, \code{\LinkA{pick\_drop}{pick.Rul.drop}}

Other functions for fgeo vft: \code{\LinkA{add\_status\_tree}{add.Rul.status.Rul.tree}},
\code{\LinkA{add\_subquad}{add.Rul.subquad}}, \code{\LinkA{guess\_plotdim}{guess.Rul.plotdim}},
\code{\LinkA{pick\_drop}{pick.Rul.drop}}
\end{SeeAlso}
%
\begin{Examples}
\begin{ExampleCode}
x <- tribble(
    ~gx,    ~gy,
      0,      0,
     50,     25,
  999.9, 499.95,
   1000,    500
)

# `gridsize` has a common default; `plotdim` is guessed from the data
add_lxly(x)

gridsize <- 20
plotdim <- c(1000, 500)

add_qxqy(x, gridsize, plotdim)

add_index(x, gridsize, plotdim)

add_hectindex(x, gridsize, plotdim)

add_quad(x, gridsize, plotdim)

add_quad(x, gridsize, plotdim, start = 0)

# `width` gives the nuber of digits to pad the label of plot-rows and
# plot-columns, e.g. 3 pads plot-rows with three zeros and plot-columns with
# an extra trhree zeros, resulting in a total of 6 zeros.
add_quad(x, gridsize, plotdim, start = 0, width = 3)

add_col_row(x, gridsize, plotdim)


# From `quadrat` or `QuadratName` --------------------------------------
x <- tribble(
  ~QuadratName,
        "0001",
        "0011",
        "0101",
        "1001"
)

# Output `gx` and `gy` ---------------

add_gxgy(x)
 
assert_is_installed("fgeo.x")
# Warning: The data may already have `gx` and `gx` columns
gxgy <- add_gxgy(fgeo.x::tree5)
select(gxgy, matches("gx|gy"))

# Output `col` and `row` -------------

# Create columns `col` and `row` from `QuadratName` with `tidyr::separate()`
# The argument `sep` lets you separate `QuadratName` at any positon
## Not run: 
tidyr_is_installed <- requireNamespace("tidyr", quietly = TRUE)
stringr_is_installed <- requireNamespace("stringr", quietly = TRUE)

if (tidyr_is_installed && stringr_is_installed) {
  library(tidyr)
  library(stringr)

  vft <- tibble(QuadratName = c("0001", "0011"))
  vft

  separate(
    vft,
    QuadratName,
    into = c("col", "row"),
    sep = 2
  )

  census <- select(fgeo.x::tree5, quadrat)
  census

  census$quadrat <- str_pad(census$quadrat, width = 4, pad = 0)

  separate(
    census,
    quadrat,
    into = c("col", "row"),
    sep = 2,
    remove = FALSE
  )
}

## End(Not run)

\end{ExampleCode}
\end{Examples}
\inputencoding{utf8}
\HeaderA{fgeo\_elevation}{Create elevation data.}{fgeo.Rul.elevation}
%
\begin{Description}\relax
This function constructs an object of class "fgeo\_elevation". It standardizes
the structure of elevation data to always output a dataframe with names \code{gx},
\code{gy} and \code{elev}.
\end{Description}
%
\begin{Usage}
\begin{verbatim}
fgeo_elevation(elev)
\end{verbatim}
\end{Usage}
%
\begin{Arguments}
\begin{ldescription}
\item[\code{elev}] One of these:
\begin{itemize}

\item A dataframe containing elevation data, with columns \code{gx}, \code{gy}, and
\code{elev}, or \code{x}, \code{y}, and \code{elev} (e.g. \code{fgeo.x::elevation\$col}).
\item A ForestGEO-like elevation list with elements \code{xdim} and \code{ydim} giving
plot dimensions, and element \code{col} containing a dataframe as described in
the previous item (e.g. \code{fgeo.x::elevation}).

\end{itemize}

\end{ldescription}
\end{Arguments}
%
\begin{Value}
A dataframe with names \code{x/gx}, \code{y/gy} and \code{elev}.
\end{Value}
%
\begin{Section}{Acknowledgments}

This function was inspired by David Kenfack.
\end{Section}
%
\begin{Examples}
\begin{ExampleCode}
assert_is_installed("fgeo.x")

# Input: Elevation dataframe
elevation_df <- fgeo.x::elevation$col
fgeo_elevation(elevation_df)

class(elevation_df)
class(fgeo_elevation(elevation_df))

names(elevation_df)
names(fgeo_elevation(elevation_df))

# Input: Elevation list
elevation_ls <- fgeo.x::elevation
fgeo_elevation(elevation_ls)

class(elevation_ls)
class(fgeo_elevation(elevation_ls))

names(elevation_ls)
names(fgeo_elevation(elevation_ls))
\end{ExampleCode}
\end{Examples}
\inputencoding{utf8}
\HeaderA{pick\_drop}{Pick and drop rows from \emph{ViewFullTable}, \emph{tree}, and \emph{stem} tables.}{pick.Rul.drop}
\aliasA{drop\_status}{pick\_drop}{drop.Rul.status}
\aliasA{pick\_dbh\_max}{pick\_drop}{pick.Rul.dbh.Rul.max}
\aliasA{pick\_dbh\_min}{pick\_drop}{pick.Rul.dbh.Rul.min}
\aliasA{pick\_dbh\_over}{pick\_drop}{pick.Rul.dbh.Rul.over}
\aliasA{pick\_dbh\_under}{pick\_drop}{pick.Rul.dbh.Rul.under}
\aliasA{pick\_status}{pick\_drop}{pick.Rul.status}
%
\begin{Description}\relax
These functions provide an expressive and convenient way to pick specific
rows from ForestGEO datasets. They allow you to remove missing values (with
\code{na.rm = TRUE}) but conservatively default to preserving them. This behavior
is similar to \code{\LinkA{base::subset()}{base::subset()}} and unlike \code{dplyr::filter()}. This
conservative default is important because you want want to include missing
trees in your analysis.
\end{Description}
%
\begin{Usage}
\begin{verbatim}
pick_dbh_min(data, value, na.rm = FALSE)

pick_dbh_max(data, value, na.rm = FALSE)

pick_dbh_under(data, value, na.rm = FALSE)

pick_dbh_over(data, value, na.rm = FALSE)

pick_status(data, value, na.rm = FALSE)

drop_status(data, value, na.rm = FALSE)
\end{verbatim}
\end{Usage}
%
\begin{Arguments}
\begin{ldescription}
\item[\code{data}] A ForestGEO-like dataframe: A \emph{ViewFullTable}, \emph{tree} or \emph{stem}
table.

\item[\code{value}] An atomic vector; a single value against to compare each value
of the variable encoded in the function's name.

\item[\code{na.rm}] Set to \code{TRUE} if you want to remove missing values from the
variable encoded in the function's name.
\end{ldescription}
\end{Arguments}
%
\begin{Value}
A dataframe similar to .\code{data} but including only the rows with
matching conditions.
\end{Value}
%
\begin{SeeAlso}\relax
\code{dplyr::filter()}, \code{Extract} (\code{[}).

Other functions for fgeo census and vft: \code{\LinkA{guess\_plotdim}{guess.Rul.plotdim}}

Other functions for fgeo census: \code{\LinkA{add\_status\_tree}{add.Rul.status.Rul.tree}},
\code{\LinkA{add\_var}{add.Rul.var}}, \code{\LinkA{guess\_plotdim}{guess.Rul.plotdim}}

Other functions for fgeo vft: \code{\LinkA{add\_status\_tree}{add.Rul.status.Rul.tree}},
\code{\LinkA{add\_subquad}{add.Rul.subquad}}, \code{\LinkA{add\_var}{add.Rul.var}},
\code{\LinkA{guess\_plotdim}{guess.Rul.plotdim}}

Other functions to pick or drop rows of a ForestGEO dataframe: \code{\LinkA{pick\_main\_stem}{pick.Rul.main.Rul.stem}}
\end{SeeAlso}
%
\begin{Examples}
\begin{ExampleCode}
census <- tribble(
  ~dbh, ~status,
     0,     "A",
    50,     "A",
   100,     "A",
   150,     "A",
    NA,     "M",
    NA,     "D",
    NA,      NA
  )

# <=
pick_dbh_max(census, 100)
pick_dbh_max(census, 100, na.rm = TRUE)

# >=
pick_dbh_min(census, 100)
pick_dbh_min(census, 100, na.rm = TRUE)

# <
pick_dbh_under(census, 100)
pick_dbh_under(census, 100, na.rm = TRUE)

# >
pick_dbh_over(census, 100)
pick_dbh_over(census, 100, na.rm = TRUE)
# Same, but `subset()` does not let you keep NAs.
subset(census, dbh > 100)

# ==
pick_status(census, "A")
pick_status(census, "A", na.rm = TRUE)

# !=
drop_status(census, "D")
drop_status(census, "D", na.rm = TRUE)

# Compose
pick_dbh_over(
  drop_status(census, "D", na.rm = TRUE), 
  100
)

# More readable as a pipiline
census %>%
  drop_status("D", na.rm = TRUE) %>%
  pick_dbh_over(100)
 
# Also works with ViewFullTables
vft <- tribble(
  ~DBH,   ~Status,
     0,   "alive",
    50,   "alive",
   100,   "alive",
   150,   "alive",
    NA, "missing",
    NA,    "dead",
    NA,        NA
)

pick_dbh_max(vft, 100)

pick_status(vft, "alive",  na.rm = TRUE)

\end{ExampleCode}
\end{Examples}
\inputencoding{utf8}
\HeaderA{pick\_main\_stem}{Pick the main stem or main stemid(s) of each tree in each census.}{pick.Rul.main.Rul.stem}
\aliasA{pick\_main\_stemid}{pick\_main\_stem}{pick.Rul.main.Rul.stemid}
%
\begin{Description}\relax
\begin{itemize}

\item \code{\LinkA{pick\_main\_stem()}{pick.Rul.main.Rul.stem}} picks a unique row for each \code{treeID} per census.
\item \code{\LinkA{pick\_main\_stemid()}{pick.Rul.main.Rul.stemid}} picks a unique row for each \code{stemID} per census. It is
only useful when a single stem was measured twice in the same census, which
sometimes happens to correct for the effect of large buttresses.

\end{itemize}

\end{Description}
%
\begin{Usage}
\begin{verbatim}
pick_main_stem(data)

pick_main_stemid(data)
\end{verbatim}
\end{Usage}
%
\begin{Arguments}
\begin{ldescription}
\item[\code{data}] A ForestGEO-like dataframe: A \emph{ViewFullTable}, \emph{tree} or \emph{stem}
table.
\end{ldescription}
\end{Arguments}
%
\begin{Details}\relax
\begin{itemize}

\item \code{\LinkA{pick\_main\_stem()}{pick.Rul.main.Rul.stem}} picks the main stem of each tree in each census. It
collapses data of multi-stem trees by picking a single stem per \code{treeid} per
\code{censusid}. From this group, it picks the stem at the top of a list sorted
first by descending order of \code{hom} and then by descending order of \code{dbh}.
This this corrects the effect of buttresses and picks the main stem. It
ignores groups of grouped data and rejects data with multiple plots.
\item \code{\LinkA{pick\_main\_stemid()}{pick.Rul.main.Rul.stemid}} does one step less than \code{\LinkA{pick\_main\_stem()}{pick.Rul.main.Rul.stem}}. It only
picks the main stemid(s) of each tree in each census and keeps all stems per
treeid. This is useful when calculating the total basal area of a tree,
because you need to sum the basal area of each individual stem as well as sum
only one of the potentially multiple measurements of each buttressed stem per
census.

\end{itemize}

\end{Details}
%
\begin{Value}
A dataframe with a single plotname, and one row per per treeid per
censusid.
\end{Value}
%
\begin{Section}{Warning}

These functions may be considerably slow. They are fastest if the data
already has a single stem per treeid. They are slower with data containing
multiple stems per \code{treeid} (per \code{censusid}), which is the main reason for
using this function. The slowest scenario is when data also contains
duplicated values of \code{stemid} per \code{treeid} (per \code{censusid}). This may
happen if trees have buttresses, in which case these functions check
every stem for potential duplicates and pick the one with the largest \code{hom}
value.

For example, in a windows computer with 32 GB of RAM, a dataset with 2
million rows with multiple stems and buttresses took about 3 minutes to run.
And a dataset with 2 million rows made up entirely of main stems took about
ten seconds to run.
\end{Section}
%
\begin{SeeAlso}\relax
Other functions to pick or drop rows of a ForestGEO dataframe: \code{\LinkA{pick\_drop}{pick.Rul.drop}}
\end{SeeAlso}
%
\begin{Examples}
\begin{ExampleCode}
# One `treeID` with multiple stems. 
# `stemID == 1.1` has two measurements (due to buttresses).
# `stemID == 1.2` has a single measurement.
census <- tribble(
    ~sp, ~treeID, ~stemID,  ~hom, ~dbh, ~CensusID,
  "sp1",     "1",   "1.1",   140,   40,         1,  # main stemID (max `hom`)
  "sp1",     "1",   "1.1",   130,   60,         1,  
  "sp1",     "1",   "1.2",   130,   55,         1   # main stemID (only one)
)

# Picks a unique row per unique `treeID`
pick_main_stem(census)

# Picks a unique row per unique `stemID`
pick_main_stemid(census)

\end{ExampleCode}
\end{Examples}
\inputencoding{utf8}
\HeaderA{read\_vft}{Import \emph{ViewFullTable} or \emph{ViewTaxonomy} data from a .tsv or .csv file.}{read.Rul.vft}
\aliasA{read\_taxa}{read\_vft}{read.Rul.taxa}
%
\begin{Description}\relax
\code{\LinkA{read\_vft()}{read.Rul.vft}} and \code{\LinkA{read\_taxa()}{read.Rul.taxa}} help you to read \emph{ViewFullTable} and
\emph{ViewTaxonomy} data from text files delivered by the ForestGEO database.
These functions avoid common problems about column separators, missing
values, column names, and column types.
\end{Description}
%
\begin{Usage}
\begin{verbatim}
read_vft(file, delim = NULL, na = c("", "NA", "NULL"), ...)

read_taxa(file, delim = NULL, na = c("", "NA", "NULL"), ...)
\end{verbatim}
\end{Usage}
%
\begin{Arguments}
\begin{ldescription}
\item[\code{file}] A path to a file.

\item[\code{delim}] Single character used to separate fields within a record. The
default (\code{delim = NULL}) is to guess between comma or tab (\code{","} or \code{"\bsl{}t"}).

\item[\code{na}] Character vector of strings to interpret as missing values. Set this
option to \code{character()} to indicate no missing values.

\item[\code{...}] Other arguments passed to \code{\LinkA{readr::read\_delim()}{readr::read.Rul.delim()}}.
\end{ldescription}
\end{Arguments}
%
\begin{Value}
A \LinkA{tibble}{tibble}.
\end{Value}
%
\begin{Section}{Acknowledgments}

Thanks to Shameema Jafferjee Esufali for inspiring the feature that
automatically detects \code{delim} (issue 65).
\end{Section}
%
\begin{SeeAlso}\relax
\code{\LinkA{readr::read\_delim()}{readr::read.Rul.delim()}}, \code{\LinkA{type\_vft()}{type.Rul.vft}}, \code{\LinkA{type\_taxa()}{type.Rul.taxa}}.

Other functions to read text files delivered by ForestgGEO's database: \code{\LinkA{type\_vft}{type.Rul.vft}}
\end{SeeAlso}
%
\begin{Examples}
\begin{ExampleCode}
assert_is_installed("fgeo.x")
library(fgeo.x)

example_path()

file_vft <- example_path("view/vft_4quad.csv")
read_vft(file_vft)

file_taxa <- example_path("view/taxa.csv")
read_taxa(file_taxa)
\end{ExampleCode}
\end{Examples}
\inputencoding{utf8}
\HeaderA{sanitize\_vft}{Fix common problems in \emph{ViewFullTable} and \emph{ViewTaxonomy} data.}{sanitize.Rul.vft}
\aliasA{sanitize\_taxa}{sanitize\_vft}{sanitize.Rul.taxa}
%
\begin{Description}\relax
These functions fix common problems of \emph{ViewFullTable} and \emph{ViewTaxonomy}
data:
\begin{itemize}

\item Ensure that each column has the correct type.
\item Ensure that missing values are represented with \code{NA}s -- not with the
literal string "NULL".

\end{itemize}

\end{Description}
%
\begin{Usage}
\begin{verbatim}
sanitize_vft(.data, na = c("", "NA", "NULL"), ...)

sanitize_taxa(.data, na = c("", "NA", "NULL"), ...)
\end{verbatim}
\end{Usage}
%
\begin{Arguments}
\begin{ldescription}
\item[\code{.data}] A dataframe; either a ForestGEO \emph{ViewFullTable}
(\code{sanitize\_vft()}).
or \emph{ViewTaxonomy} (\code{sanitize\_vft()}).

\item[\code{na}] Character vector of strings to interpret as missing values. Set this
option to \code{character()} to indicate no missing values.

\item[\code{...}] Arguments passed to \code{\LinkA{readr::type\_convert()}{readr::type.Rul.convert()}}.
\end{ldescription}
\end{Arguments}
%
\begin{Value}
A dataframe.
\end{Value}
%
\begin{Section}{Acknowledgments}

Thanks to Shameema Jafferjee Esufali for motivating this functions.
\end{Section}
%
\begin{SeeAlso}\relax
\code{\LinkA{read\_vft()}{read.Rul.vft}}.
\end{SeeAlso}
%
\begin{Examples}
\begin{ExampleCode}
assert_is_installed("fgeo.x")

vft <- fgeo.x::vft_4quad

# Introduce problems to show how to fix them
# Bad column types
vft[] <- lapply(vft, as.character)
# Bad representation of missing values
vft$PlotName <- "NULL"

# "NULL" should be replaced by `NA` and `DBH` should be numeric
str(vft[c("PlotName", "DBH")])

# Fix
vft_sane <- sanitize_vft(vft)
str(vft_sane[c("PlotName", "DBH")])

taxa <- read.csv(fgeo.x::example_path("taxa.csv"))
# E.g. inserting bad column types
taxa[] <- lapply(taxa, as.character)
# E.g. inserting bad representation of missing values
taxa$SubspeciesID <- "NULL"

# "NULL" should be replaced by `NA` and `ViewID` should be integer
str(taxa[c("SubspeciesID", "ViewID")])

# Fix
taxa_sane <- sanitize_taxa(taxa)
str(taxa_sane[c("SubspeciesID", "ViewID")])
\end{ExampleCode}
\end{Examples}
\printindex{}
\end{document}
